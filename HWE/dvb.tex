\documentclass[a4paper]{article}
\usepackage[utf8]{inputenc}
\usepackage[german]{babel}

\begin{document}

\section{DVB}

\subsection{Allgemeines}

DVB steht für "Digital Video Broadcast" und beschreibt ein standardisiertes Verfahren zur Übertragung von digitales Fernsehen. Durch aufwendige Datenkompression (z.B. MPEG-4) können heutzutage im Gegensatz zum analogen Fernsehen durch höhere Datenkompression mehr Programme pro Sendekanal übertragen werden.

\subsection{Übertragungsmedien}

\begin{itemize}
\item \textbf{DVB-S} $\rightarrow$ für die Übertragung durch direkt strahlende Satelliten 
\item \textbf{DVB-S2} $\rightarrow$ aktueller Nachfolgestandard für DVB-S
\item \textbf{DVB-C} $\rightarrow$ für die Übertragung über Kabelnetze
\item \textbf{DVB-C2} $\rightarrow$ Nachfolger des DVB-C
\item \textbf{DVB-T} $\rightarrow$ für die Übertragung durch terrestrische Senderketten im VHF- bzw. UHF-Bereich
\item \textbf{DVB-T2} $\rightarrow$ Nachfolger des DVB-T Standards
\item \textbf{DVB-H} $\rightarrow$ für die asynchrone Übertragung auf mobile Endgeräte, ebenfalls terrestrisch
\item \textbf{DVB-IPI} $\rightarrow$ für die Übertragung über IP-basierende Netzwerke, zum Beispiel Internet (Internet Protocol Infrastructure)
\item \textbf{DVB-RC(S/C/T} $\rightarrow$ Rückkanal (Return Channel) für die Übertragung von Datendiensten, zum Beispiel Breitbandinternet
\item \textbf{DVB-SI} $\rightarrow$ für die Übertragung der Service Information
\item \textbf{DVB-SH} $\rightarrow$ für die Übertragung über Satellit auf mobile Endgeräte
\end{itemize}

\subsection{Unterschiede bei der Übertragung bei verschiedenen Übertragungswegen}

\begin{itemize}

	\item \textbf{DVB-S (Satellit)} $\rightarrow$ \begin{itemize}
	\item Modulationsart: QPSK
	\item Übertragungskapazität: typsich 33Mbit/s bis 38Mbit/s
	\item Empfang: Parabolantenne 
	\item Mobilität: stationär, bedingt tragbar (mobil)
	\end{itemize}
	
	\item \textbf{DVB-S2 (Satellit, HDTV)} $\rightarrow$ \begin{itemize}
	\item Modulationsart: QPSK, 8PSK, 16APSK oder 32APSK
	\end{itemize}	
	
	\item \textbf{DVB-C (Kable)} $\rightarrow$ \begin{itemize}
	\item Modulationsart: 16 bis 256 QAM
	\item Übertragungskapazität: typsich 38Mbit/s (64 QAM)
	\item Mobilität: stationär
	\end{itemize}

	\item \textbf{DVB-C2 (Kable)} $\rightarrow$ \begin{itemize}
	\item Modulationsart: 16 bis 4096 QAM
	\item Übertragungskapazität: typsich 38Mbit/s (64 QAM), 83 Mbit/s (4096 QAM)
	\item Mobilität: stationär
	\end{itemize}
	
	\item \textbf{DVB-T (Terrestrisch)} $\rightarrow$ \begin{itemize}
	\item Modulationsart: QPSK, 16-QAM, 64-QAM
	\item Übertragungskapazität: typsich 4Mbit/s bis 22Mbit/s
	\item Mobilität: stationär, tragbar, mobil
	\end{itemize}



\end{itemize}





\end{document}
\documentclass[a4paper]{article}

\usepackage[utf8]{inputenc}

\title{Das Epische Theater}
\author{Alin Porcic}

\begin{document}

	\maketitle
	\newpage

        \section{Bertolt Brecht (1898 - 1956)}

	Bertolt Brecht wurde am 10.Febuar 1898 in Augsburg, Deutschland, geboren und wuchs in gesicherten ökonomischen und sozialen Verhältnissen auf. Sein Vater hatte keine höhere Schulausbildung, konnte sich aber in der Augsburger Haindl'schen Papierfabrik zum Direktor hoch arbeiten.\\\\
        Bertolt Brecht besuchte stadesgemäß das Realgymnasium und brachte regelmäßig gute, wenn nicht sehr gute Zeugnisse nach Hause. In seiner Freizeit besuchte er einen Klavier-, Geigen- und Gitarrenunterricht.\\\\
        Schon im jungen Alter began Bertolt mit dem Dichten und arbeitet zusammen mit seinen Freunden an einer Schülerzeitung. Er verfasste Gedichte, Prosatexte und sogar ein einaktiges Drama, die Bibel. In den folgenden Jahren produzierte er weitere Gedichte und Dramenentwürfe.\\\\
        Nach dem Beginn des Ersten Weltkrieges 1914 gelang es ihm, eine Serie von Reportagen und Rezensionen in die lokalen und reginalen Medien unterzubringen (München-Augsburger Abendzeitung). 1916 verfasste er Gedichte, die in der späteren Sammlung 'Bertolt Brechts Hauspostille' eingebunden wurden.\\\\
        Im März 1917 meldete sich Brecht zum Kriegshilfedienst und erlangte so die Genehmigung für ein vereinfachtes Notabitur (oder auch Kriegsmatura). Seinen Dienst leistet er mit Schreibarbeiten sowie Gärnterei ab. Später wurde er vom Kriegsdienst zurückgestellt, arbeitet als Hauslehrer und began ein Studium in München.\\\\
        Bertolt studierte Medizin und Philosophie, jedoch konzentrierte er sich auf die Seminare von Artur Kutscher zum Thema Gegenwartsliteratur. Dort lernte er den von ihm bewunderten Lyriker und Dramatiker Frank Wedekind sowie Otto Zarek und Hanns Johst kennen.\\\\
	In den ersten zwie Semestern war es Brecht mit der Unterstützung seines Vater gelungen, eine Zurückstellung vom Militärdienst zu erreichen, jedoch wurde er im Oktober 1918 in das Augsburger Reservelazarett einberufen.\\\\
        Nach der Novemberrevolution war Bertolt Mitglied des Lazarettrates und auch des Augsburger Arbeiter- und Soldatenrats. Am 9 Januar 1919 konnte er seinen Dienst schon wieder beendet.\\\\
        Brecht verfasste ein Drama names 'Spartakus', welches jedoch später in 'Trommeln in der Nacht' umbenannt wurde. Um 1920 verfasste Brecht einige Einakter, unter danderem 'Die Hochzeit' (später betitelt: Die Kleinbürgerhochzeit), die unaufgeführt blieben. Er schrieb Theaterkritiken für die Augsburger USPD-Zeitug 'Der Volkswille'.\\\\
        Nach dem Kapp-Putsch 1920 kehrte Bertolt Brecht erfolglos nach München zurück. 1921 gelang es Bertolt 
        
        \section{Das Epische Theater}

		\subsection{Allgemeines}
                
        	\subsection{Verfremdungseffekte}

		\subsection{Nennenswerte Autoren und Werke}
        
        \section{Mutter Courage}

\end{document}

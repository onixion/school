\documentclass[a4paper]{article}

\usepackage[utf8]{inputenc}
\usepackage{graphicx}

\parindent 0px

\title{Das Epische Theater}
\author{Alin Porcic}

\begin{document}

	\maketitle

	\newpage

	\section{Das Epische Theater}

        Das Epische Theater, ein von Bertolt Brecht geprägter Begriff, verbindet die zwei literarische Gattungen: die Dramatik und die Epik. In den 1920er Jahren experimentierten Bertolt Brecht und Erwin Piscator an einem spezeziellen  Drama. Sie versuchten die Darstellung von tragischen Einzelschicksalen auf der Bühne zu vermeiden und versuchten stattdessen gesellschaftliche Konflikte, wie zum Beispiel Kriege, Revolutionen oder soziale Ungerechtigkeit, durchschaubar aufführbar zu machen und den Zuseher dazu bewegen, die Gesellschaft zum Positiven zu verändern. Nach Brecht soll das Epische Theater durch die Demonstartation von gesellschaftliche und politische Widersprüche positive Veränderungen in Gang setzten.\\\\

        Der Zuschauer soll das Geschehen kritisch und emotionslos betrachten und sich dabei von den gezeigten Ereignissen distanzieren. Das Mitgefühl der Zuschauer zu den Figuren soll unterbunden werden und der Zuschauer soll gesellschaftliche Erkenntnisse sammeln. Im Gegensatz zum aristoteleschen Theater hat das Epische Theater keinen Spannungbogen, jede Szene steht für sich und kann in der Reihenfolge verändert werden. Ein offener Schluss ist im Epischen Theater eher üblich.\\\\

	Die Verfremdungseffekte sind stilistische Methoden, die das Epische Theater anwendet, um die Entstehung von Spannung zu vermeiden. Bertolt Brecht ist der Meinung, dass die Spannung den Zuschauern im Denken und Lernen behindere und deshalb entfernt werden müsse. Das wird durch folgende Methoden erreicht:
        
        \begin{itemize}
	\item Der Szeneninhalt wird schon vor der Aufführung der jeweiligen Szenen erzählt. Damit weiß der Zuschauer was ihm erwartet und er kann sich ganz auf den Inhalt des Stücks konzentieren.
        \item Das Bühnenbild wird gezielt verändert, sodass der Zuschauer immer weiß, dass er sich im Theater befindet und die Ereignisse nicht real sind.
        \item Kommentare unterbrechen die Handlungen, Figuren treten aus ihren Rollen aus und wenden sich an das Publikum
        \item Es wird zum Teil in Versen gesprochen.
        \item Die Schauspieler selbst bauen eine Distanz zur gespielten Figur auf, damit die Zuschauer den Protagonisten nicht als Identifikationsfigur wahrnehmen können. Die Figuren können kritische betrachtet werden.
        \end{itemize}
        
        \newpage
        \section{Bertolt Brecht (1898 - 1956)}

	Bertolt Brecht wurde am 10.Febuar 1898 in Augsburg, Deutschland, geboren und wuchs in gesicherten ökonomischen und sozialen Verhältnissen auf. Sein Vater hatte keine höhere Schulausbildung, konnte sich aber in der Augsburger Haindl'schen Papierfabrik zum Direktor hoch arbeiten.\\\\
        Bertolt Brecht besuchte stadesgemäß das Realgymnasium und brachte regelmäßig gute, wenn nicht sehr gute Zeugnisse nach Hause.\\\\
        Schon im jungen Alter began Bertolt mit dem Dichten und arbeitet zusammen mit seinen Freunden an einer Schülerzeitung. Er verfasste Gedichte, Prosatexte und sogar ein einaktiges Drama, die Bibel. In den folgenden Jahren produzierte er weitere Gedichte und Dramenentwürfe.\\\\
        Nach dem Beginn des Ersten Weltkrieges 1914 gelang es ihm, eine Serie von Reportagen und Rezensionen in die lokalen und reginalen Medien unterzubringen (München-Augsburger Abendzeitung). 1916 verfasste er Gedichte, die in der späteren Sammlung 'Bertolt Brechts Hauspostille' eingebunden wurden. 1920 arbeitet er mit Erwin Piscator am Epischen Theater und verfasst weiter Dramen.\\\\

	\subsection{Werke}

        \begin{itemize}
        \item Baal
	\item Trommeln in der Nacht
	\item Mutter Courage und ihre Kinder
        \item Der gute Mensch von Sezuan
        \item Die Mutter
        \item Der Brotladen
        \item Die sieben Todsünden
        \item Die Maßnahme
        \item ...
          
        \end{itemize}
        
        \section{Mutter Courage und ihre Kinder}

	'Mutter Courage und ihre Kinder' ist ein Drama, welches von Bertolt Brecht im schwedischen Exil 1938/39 verfasst wurde und 1941 in Zürich uraufgeführt wurde. Im dramatischen Werk geht es um die Mutter Courage, die versucht mit dem Krieg Geschäfte zu machen und dabei alle drei Kinder verliert. Bertolt Brecht möchte mit diesem Werk Abscheu vor dem Krieg vermitteln und vor der kapitalistischen Gesellschaft.
        
	\subsection{Inhaltsangabe}        

	Anna Fierling, auch Mutter Courage genannt, zieht mit ihrem Marktwagen, ihren beiden Söhnen, dem mutigen Eilif, dem ehrlichen, aber dummen Schweizerkas und ihrer stummen Tochter Kattrin durch die Lande.\\
        In Südschweden wird Eilif von einem Feldwebel für den Krieg gewornem. Die sehr pessimistisch gestellte Mutter Courage sagt dem Feldwebel den Tod voraus, aber auch, dass ihre eigenen Kinder den Tod finden werden. Zwei Jahre später sieht sie ihren Sohn Eilif als Held in Polen wieder. Seine Heldentat, er hat einem Bauern das Vieh gestohlen, belohnt sie mit einer Ohrfeige. Gemeinsam mit einem finnischen Regimetn gerät Mutter Courage in Gefangenschaft der Katholiken. Als Schweizerkas die Regimentskasse in Sicherheit bringen will, wird er ertappt und vor dem Feldgericht verurteilt. Um ihn auslösen zu können, verpfändet Mutter Courage ihren Wagen, doch sie feilscht so lange, bis Schweizerkas erschossen wird. Als ihre Waren mutwillig zerstört werden, möchte sie sich beim Rittmeister beschweren, doch sie besinnt sich, denn es ist ihrer Meinung nach besser, im Krieg Handel zu betreiben, als Gerechtigkeit zu suchen. Ein protestantischer Feldprediger hilft ihr, sich dem katholischen Heer anzuschließen. Aufgrund eines ÜBerfalls auf Kattrin, wechselt Mutter Courage die Front. Eilif wird zum Tode verurteilt, weil er eine Bauersfrau umgebracht hat. Vier Jahre vergehen und Kattrin belauscht das Gespräch einiger kaiserlichen Soldaten, die die Stadt Halle strümen wollen und steigt auf das Dach des Hauses, um die Bewohner zu warnen. Es gelingt ihr auch, doch sie wird von einem Soldaten erschossen. Mutter Courage zieht mit ihrem Wagen alleine weiter. Sie hat alle drei Kinder verloren und nichts aus dem Krieg gelernt.
        
        \subsection{Figurenkonstallation}
	\includegraphics[width=250px]{img/figuren.png}
        
\end{document}

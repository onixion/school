\documentclass[12pt,a4paper]{article}

\usepackage[german]{babel}
\usepackage[utf8]{inputenc}

\title{Literatur}
\author{Alin Porcic}

\begin{document}

\maketitle
\newpage
\tableofcontents
\newpage

\section{Literarische Gattungen}
\subsection{Lyrik}

Mit der Lyrik ist die liedhafte Dichtung gemeint, die meistens strophenförmig aufgebaut ist. Sie kann auf zwei verschiedene Möglichkeiten gereimt sein:
	\begin{itemize}
	\item Stabreim: gleicher Anlaut z.B. Mann und Maus
	\item Endreim: Gleichklang ab der letzen betonten Silbe
	\end{itemize}

\subsection{Dramatik}

Die Dramatik ist eine sehr alte literarische Gattung, die speziell für die Bühne gedacht ist. Die Ereignisse spielen in der Gegenwart und der Inhalt kann im Monolog und Dialog wiedergegeben werden. Das Drama wird durch Bühnenbild, Akustik, Gestik, Mimik etc. verstärkt.\\
\newline
\newline
Die Regeln des Aristoteles:

	\begin{itemize}
	\item Große Fallhöhe der Figuren
	\item Drei Einheiten müssen gewahrt werden
		\begin{itemize}
		\item Einheit des Ortes (kein Ortswechsel)
		\item Einheit der Zeit (das Stück darf nich länger als einen Tag dauern)
		\item Einheit der Handlung (es darf nicht mehrere Handlungsstänge geben)
		\end{itemize}
	\item Kataris (die Zuschauer dazu bringen, nicht die gleichen Fehler wie im Stück zu machen)
	\item Hochsprache muss verwendet werden
	\end{itemize}

Das Drama hat zwei Formen:

	\begin{itemize}
	\item Geschlossene Drama
		\begin{itemize}
		\item wenig Figuren
		\item Handlung auf das Ende hin einstränig
		\item fester Schluss
		\end{itemize}
	\item Offenes Drama
		\begin{itemize}
		\item viele Figuren
		\item mehrere Handlungsstänge
		\item offenes Ende
		\end{itemize}
	\end{itemize}
	
Das klassische Drama von Aristoteles hat fünf Akten:

	\begin{itemize}
	\item 1.Akt (Einleitung)
	\item 2.Akt (Steigerung der Verwicklung)
	\item 3.Akt (Höhepunkt)
	\item 4.Akt (Umschlag und fallende Handlung)
	\item 5.Akt (Katastrophe, Rettung, Lösung; ist abhängig vom Dramentyp)
	\end{itemize}
	
Dramentypen:

	\begin{itemize}
	\item Tragödie: Endet mit dem Untergang
	\item Komödie: Löst die Verwicklung unter Belustigung auf
	\item Schauspiel: Held geht nicht unter / Mittellage
	\end{itemize}

\subsection{Epik}

Die Epik ist die erzählende Dichtung. Es taucht ein Erzähler auf, der die Geschichte vorträgt und dabei kann er verschiedene Erzählverhalten aufweisen:

	\begin{itemize}
	\item auktoriales Erzählverhalten: Der Erzähler hat die olympische (göttliche) Position und kann gelegentlich in der Ich/Er-Form kommentierend eingreifen. Zudem weiß er die Zukunft und das Ende der Handlung.
	\item neutrales Erzählverhalten: Der Erzähler ist ein Feldherr, er hat einen guten Überblick, jedoch schlechte Detailkenntnisse über die Lage.
	\item persönliches Erzählverhalten: Der Erzähler ist selbst am Geschehen  beteiligt, weiß die Gefühle seiner Figur und hat im Gegensatz zum neutralem Erzählverhalten bessere Detailkenntnisse.
	\end{itemize}
	
Der Erzähler kann verschiedene Erzählhaltungen haben:

	\begin{itemize}
	\item kritisch
	\item ironisch
	\item neutral
	\item ...
	\end{itemize}

Epische Texte sind gekennzeichnet durch:

	\begin{itemize}
	\item Inhalt: äußeres Gerüst einer Geschichte (Handlungsverlauf und Figurenkonstalation; Inhaltsangabe)
	\item Fabel: Inhalt auf das Äußerste reduziert (wenig Sätze)
	\item Stoff: Erlebnis, Ergebnis die dem Autor zum Schreiben angeregt haben
	\item Stillage: Art und Weise der sprachlichen Darstellung 
	\end{itemize}	
	
Epische Textsorten:

	\begin{itemize}
	\item Roman: Es gibt kein Merkmal, dass für alle Romane zutrifft, aber meistens erzählen Romane eine mehrsträngige Geschichte über
	einen längeren Zeitraum mit vielen Figuren. Der Begriff Roman heißt übersetzt romanisch und bedeutet Volkssprache. Es gibt viele verschiedene
	Romane z.B. Liebesroman, Thrillerroman, Science-Fiction-Roman, Abenteuerroman usw.
	\item Erzählung: Die Erzählung beschränkt sich auf einen engeren Weltausschnitt. Sie hat wenig Figuren, jedoch sind die erzählten Ereignisse in
	der realen Welt umsetzbar.
	\item Kurzgeschichte: Kurzgeschichten haben eine geringe Zeitspanne, die Figuren werden nicht entwickelt, sondern kommen in einem krisenhaftem
	Moment vor und das prägt ihr auftreten. Ihr Charakter ergibt sich aus ihrem handeln und ihren aussagen. Hierbei wird einfache knappe Sprache 
	verwendet und der Schluss ist meistens offen.
	\item Novelle: Die Novelle kommt aus dem Italienischen und heißt Neuigkeit (Zeit: Renaissance). Sie handeln von besonderen Ereignissen, die nicht
	alle Tage vorfallen. Aufbau einer Novelle:
		\begin{itemize}
		\item Rahmenerzählung = nennt den Erzählanlass
		\item Binnengeschichte = erzählt die eigentliche Geschichte
		\end{itemize}
	\item Epos: Das Epos kommt aus dem Griechischen und wird meistens in Fersen kunstvoll gestaltet. Sie handeln meistens um dramatische 
	Ereignisse z.B. Kriege, Leidenschaften. Einer der bekanntesten Dichter ist Homer mit der Ilias und Odysee.
	\end{itemize}
	
\newpage	
	
\section{Mittelalter}	

Es herrscht das Lehenswesen und der Klärus ist für die Bildung zuständig.
Das Nittelalter ist die Blütezeit des Rittertums und zeichnetet sich durch romantische und gotische Baustile aus.
Nur die wenigsten konnten schreiben (Mönche, Adelige, ...).	

\subsection{Epen aus dem Mittelalter:}

	\begin{enumerate}
		\item Volks- und Heldenepos (z.B. Nibelungenlied):
			\begin{itemize}
			\item Erzählungen aus der Völkerwanderung (Heldenlieder, Sagen, ...)
			\item Strophen
			\item vier Langzeilen mit Paarreim
			\item Zäsur (Einschnitt)
			\item lange Zeit nur mündlich überliefert
			\item Vermischung folgender Sagenkreise bei dem Nibelungenlied:
			
				\begin{enumerate}
				\item Sage um Siegfried - Kampf gegen den Drachen
				\item Sage um Brünhild von Island
				\item Herrscherfamilie in Worms
				\item Attila der Hunnenkönig
				\end{enumerate}
			
			\end{itemize}
			
		\item Höfisches Epos (=das höfische Epos):
			\begin{itemize}
			\item keine Strophen
			\item Paarreim
			\item Kurzzeilen
			\item keine Zäsur
			\item handelt von fiktiven Rittergeschichten
			\end{itemize}						
	
		Bekanntesten höfische Epen:
		
			\begin{itemize}
			\item Der arme Heinrich und Iwein, geschrieben von Hartmann von Aue, handelt von drei Rittern Namens 						Heinrich, Erec und Iwein. Ein Ritter soll freigiebig sein, Ehre
			haben, erbarmen mit armen Menschen haben, muss verlässlich sein und dürfen nicht zu faul sein. Erec und
			Iwein sind zu faul und verlieren ihren Platz an der Tafelrunde.
			
			\item Parzival, geschrieben von Wolfgang von Eschenbach, handelt von dem heiligen Gral.				
			
			\item Tristan und Isolde (Gottfried von Stoßburg)
			\end{itemize}				
	
		\item Minnegesang: Unter Minne versteht man die Frauenverehrung. Es gibt hohe Minne und die niedere. Mit 
		der niederen Minne geht es um den Sex und Liebesspiele. Mit ''frouwe'' bezeichnet man die adelige Herrin und
		''wib'' (=Weib) bezeichnete 	damals neutral die Frau.
		
		\item Politische Sprüche
	
\end{enumerate}	
	
\subsection{Laut- und Bedeutungswandler vom Mittelhochdeutsch zu dem Neuhochdeutsch}

\subsubsection{Lautwandel}
Um ca. 600 n.C. verschob sich die Laute (Lautverschiebung). Harte Verschlusslaute wie z.B. P, T und K wurden weicher $\rightarrow$ P wurde zu PF, T zu S oder SS und K zu CH.

\begin{itemize}
\item Diphthongierung (12/13 Jahrhundert): einfache Vokale wurden zu Zwielauten; ausgehened vom Mitteldeutschland nach
Süden, nicht erfasst wurde Norddeutschland und der Alemannischer Raum; z.B. Hus wird zu Haus, Mus wird zu Maus, wib wird
zu Weib, min wird zu mein, ...

\item Monothongierung (13 Jahrhundert): Zwielaute werden wieder zu einfachen Vokalen; nicht erfasst wurde der süddeutsche Raum siehe Tiroler Dialekt
\end{itemize}

\begin{itemize}
\item Assimilation (=Angleichung an benachbarten Lauten): z.B. Fibel-Bibel, Zimber-Zimmer
\item Dissimilation (=gleicher Buchstabe wird unterschiedlich oder fällt weg): z.B. Turtur-Turtel,Pfenning-Pfennig
\end{itemize}
	
\subsubsection{Bedeutungswandel}
	
\begin{itemize}
\item Bedeutungsverschlechterung: wib - Weib,frowe - Frau
\item Bedeutungsverbesserung: marschalk - Marschall
\item Bedeutungsverengung: hochgezite - Hochzeit
\item Bedeutungserweiterung: vertec - fertig

\end{itemize}

\newpage

\section{Renaissance}

Die Renaissance (=Wiedergeburt der Antike) ist die Wiedergeburt der Antike (300 v.C. bis 300 n.C.) in den Bereichen
Kunst und Literatur. Kernzeitraum der Renaissance ist 14/15/16 Jahrhundert. Zentrum der Renaissance war der Hof der
Medici in Florenz. Wesentliche Strömung war der Humanismus.

\subsection{Literatur in der Renaissance}

\begin{itemize}
\item Dante Alighieri - Göttliche Komödie
\item Giovanni Boccaccio - Novellensammlung ''Decamerone''\\
 Bekannteste Novelle: Falkennovelle 
\end{itemize}

\subsection{Berühmte Künstler dieser Epoche}

Italienische Künstler:

\begin{itemize}
\item Raffael
\item Leonard da Vinci
\item Bromante
\item Michelangelo
\end{itemize}

Deutsche Künstler:

\begin{itemize}
\item Albrecht Dürer (z.B. der Hase)
\item Lukas Cranach
\end{itemize}

\subsection{Humanismus}
Übersetzt bedeutet Humanismus Menschlichkeit und orientiert sich an der Würde und den Werten des einzelnen Menschens.
Grundsetzte sind Toleranz, Gewaltfreiheit und Gewissenfreiheit. Die Fragen mit denen sich der Humanismus befasst: Was
ist der Mensch und was ist sein wahres Wesen. Der Humanismus läuft parallel zur Aufklärung.

\subsection{Reformation (1517)}
Martin Luther schlägt bei Schlossberg von Wittenberg 95 Thesen gegen den Papst und das führt zu einem Erdbeben in 
der Kirchengeschichte. In Thüringen finden Martin einen Unterschlumpf und übersetzt dort die Bibel in den thüringer 
Dialekt. Dies führt zu der Spaltung der Religion in Katholisch und Evangelisch.

\newpage

\section{Barock}
Die Barockepoche beginnt 1618 und endet mit dem dreißigjährigen Krieg 1648 (17 Jahrhundert).
Der barocker Baustil vernichtet ältere Baustile. Typische Barock ist viel Gold, übertriebene Bauten und üppige Figuren.

Bekannte Maler:

\begin{itemize}
\item Peter Paul Rubens (Holländer)
\item Rembrandt: bekannt durch Lichtquellen (Schatten usw.)
\end{itemize}

\subsection{Literatur im Barock}
\begin{itemize}
\item Andreas Gryphius: Lyrik $\rightarrow$ ''Alles ist eitel.''
\item Grimmelshausen: Simplicissimus (Schelmenroman) $\rightarrow$ ist eine Lebensbeschreibung des Autors.
\end{itemize}

\newpage

\section{Aufklärung}

Die Aufklärung (1700 bis 1770) war eine geistliche Strömung, die sich hauptsächlich mit dem Thema Bildung beschäftigte. Nach Emanuel Kant ist die Definition der Aufklärung ''der Ausgang des Menschens aus der selbstverschuldeten Unmündigkeit (sich selber zu bilden)''. Das oberste Ziel der Aufklärung war die Verbreitung der Bildung (Schulpflicht), dabei wurde das Menschenrecht überdacht und es wurde versucht den Menschen zu einer freien, von der Vernuft geleiteten, Persönlichkeit zu verleiten.\\

Es gibt zwei unterschiedliche Hauptströme während der Aufklärung:

	\begin{itemize}
	\item Rationalismus: Der Hauptvertreter des Rationalismus war Rene Descartes, der davon ausging, das überlieferte Wissen nicht einfach hinzunehmen ist, sondern alles in Zweifel zu ziehen. Dabei fand er heraus das die einzige Erkenntnis die Unzweifelbar ist ''Ich denke, also bin ich.''. Damit machte der Rationalismus die Vernuft zur einzigen Erkenntnissquelle für wahr und falsch.
	
	\item Empirismus: Der Hauptvertreter des Empirismus war John Locke, der die Beobachtung zur Grundlage wissenschaftlichen Aussagen macht. Er meint, Wissen bilde sich allein aus unserer Wahrnehmung und Beobachtungen.
	\end{itemize}

\subsection{Die Entwicklung des deutschen Dramas während der Aufklärung}

\begin{itemize}
\item Christoph Gottsched: Gottsched wollte das das deutsche Drama sich nach dem französischen Drama richtete, da das englische Drama nach Shakespeares ihn seinen Augen ein Gräuel war. 


\item Gotthold Ephraim Lessing: Für Lessing sollte das deutsche Theater sich nach Shakespeares richten, da er sich nicht an die strengen Regeln des Aristoteles halten wollte. Lessing verfasste die Ringparabel (=Gleichnis) ''Nathan der Weise''.
\end{itemize}

\newpage

\section{Sturm und Drang}

Mit Sturm und Drang wird eine Epoche in der Deutschen Literatur von 1770 bis 1785 bezeichnet. Es tritt ein junges Kraftgenie auf. Dieser ist ein junger und intelligenter Anführer. Die Natur wird nicht mehr als Bedrohung gesehen, sondern als Spiegel zur Seele.

\subsection{Briefroman}

Der Briefroman ist eine Sonderform des Romans, der als Abfolge bzw. Wechsel von Briefen von einem oder mehrere Korrespondenzen verfasst ist. Diese Briefe werden noch durch andere autobiografische Zeugnisse (Tagebucheinträgen) oder Kommentaren des Herausgebers ergänzt. Empfindsame Briefe und Briefwechsel waren für das Lesepublikum ein fazinierendes Medium der Gefühlserkundung.\\
Eigenschaften des Briefromans:

\begin{itemize}
\item wird in der Ich- oder Er-Form geschrieben
\item beliebige Zeit- und Ortssprünge sind möglich
\item sie erlauben unmittelbare Anteilnahme der Leser am Geschehen
\item der Leser neigt dazu, dass es wirklich so passiert ist (Echtheitscharakter)
\end{itemize}

\subsubsection{Leiden des jungen Werthers}

Der Briefroman ''Leiden des jungen Werthers'', geschrieben von Goethe, handelt von dem jungem Werther, der sich in eine Frau verliebt, die schon vergeben ist. Der Werther schreibt seinen besten Freund Willhelm, der jedoch nicht auf seine Briefe antwortet. In den eizelnen Briefen spiegeln sich die Gefühle des Werthers in der Natur (Werther traurig $\rightarrow$ Regen). Am Ende des Roman nimmt der Werther sich das Leben, da er die Frau für das Leben nicht bekommt.

\subsubsection{Die Räuber}

Das Drama ''Die Räuber'',geschrieben von Schiller, handelt von zwei Brüdern, die eine Räuberbande gründen, um den reichen Menschen das Geld zu stehlen und es den Armen zu geben (Robin Hood Motiv). Karl Moore ist im Stück das junge Kraftgenie. Das Drama endet damit, dass sich Karl Moore für die Taten der Bande verantworten muss.

\newpage

\section{Deutsche Klassik}

Die Epoche der Deutschen Klassik beginnt mit Goethes erster Italienreise 1786 und endet mit Schillers Tod 1805. Das Vorbild der
Deutschen Klassik ist die Antike.\\

Eigenschaften des Deutschen Klassik:

\begin{itemize}
\item Gott als sittliche Weltordnung
\item sein sittlicher Wille (=Mensch entscheidet frei und ist nicht von Göttern abhängig wie in der Antike)
\item die in sich ruhende Persönlichkeit
\item Objektivismus
\item überkonfessionell
\item übernational (Nationalität spielt keine Rolle)
\item Reinheit der Gattungen
\item Kunstdichtung
\item Vollendete (hält sich an die Regeln)
\item Vorlieben für strengen Aufbau
\item geschlossene Form
\item Darstellung des Allgemeinmenschlichen im Einzelschicksal
\end{itemize}

\subsection{Werke zur Deutschen Klassik}

\begin{itemize}

\item Goethe
	\begin{itemize}
		\item Lyrik:
			\begin{itemize}
			\item West-Östlicher Divan
			\end{itemize}					
				
		\item Drama:
			\begin{itemize}
			\item Iphigenie auf Tauris
			\item Faust 1 und 2 (Teufelsbündnis)
			\end{itemize}
			
		\item Epic:
			\begin{itemize}
			\item die Wahlverwandschaften
			\end{itemize}
			
		\item Ballade:
			\begin{itemize}
			\item Zauberlehrling
			\item der Erlkönig
			\end{itemize}
	
	\end{itemize}
	
\item Schiller
		\item Lyrik:
			\begin{itemize}
			\item lol
			\end{itemize}					
				
		\item Drama:
			\begin{itemize}
			\item Wilhelm Dell
			\item Jungfrau von Orleane
			\item Wollenstein
			\end{itemize}
			
		\item Epic:
			\begin{itemize}
			\item die Wahlverwandschaften
			\end{itemize}
		
		\item Ballade:
			\begin{itemize}
			\item die Bürgschaft
			\item die Glocke
			\item der Taucher
			\item der Handschuhe
			\end{itemize}
	
\end{itemize}

\subsection{Iphigenie auf Tauris (Goethe)}

Goethe hat das Stück ''Iphigenie auf Tauris'' von Euripides neu aufgefasst und dabei einige Dinge geändert:

\begin{itemize}
\item Orest hat die Aufgabe seine Schwester mit nach Hause zu nehmen (nach Euripides soll Iphigenie eine Statue der Königin Tiana mitnehmen)
\item Iphigenie bittet den Köning von Tauris sie freizulassen (in der euripischen Fassung flüchtet Iphigenie) (die in sich ruhende Persönlichkeit, Ehrlichkeit)
\end{itemize}

\newpage

\section{Romantik}

Die Romantik (1795 bis 1835, parallel zur Klassik) ist die Gegenströmung zur Klassik.\\

Eigenschaften der Romantik:

\begin{itemize}
\item Gott als gefühl für das Unendliche
\item Wichtige Fähigkeiten des Menschens sind Phantasie und Sehnsucht
\item Subjektivismus
\item national
\item Volksdichtung
\item Vermischung der Gattungen
\item Vorlieben für lockeren Aufbau
\item offene Form
\item Darstellung des Individuellen
\end{itemize}

\subsection{Dreieilung der Romantik}

\subsubsection{Frühromantik}

Die Frühromantik kann geografisch auf einer Stadt zugeordnet werden: Jena. Jena war eine deutsche Universitätsstadt im heutigen Thüringen. Dort schrieben berühmte Autoren wie zum Beispiel Tieck und Novalis ihre Bücher (Künstlerromane). Für sie spielten die Fantasy eine wichtige Rolle.

\subsubsection{Hochromantik}

Heidelberg ist eine Großstadt im süd-westen Deutschland und wird hauptsächlich von jungen Leuten vertreten (Studenten). Es wird gezielt nach alten Schriften gesucht (Kloster) und nach mündlichen Schriften.

\begin{itemize}
\item Clemens Brentano: des Knaben Wunderhorn, Sammulung von Deutschen Liedern
\item Achim von Arnim
\item Gebrüder Grimm: sammelten Volksmärchen, erstes Wörterbuch wird entwickelt
\end{itemize}

\subsubsection{Berlinerromantik}

\newpage

\section{Biedermeier = Vormärz (1915 bis 1948)}

\subsection{Wichtiger Ereignisse}

	\begin{itemize}
	\item 1915 wird in Waterloo besiegt und auf die Insel St.Helena verbannt
	\item 18.September.1915 findet der Wiener Konkress statt. Dort werden neue Grenzen und Regeln festgelegt und es werden die alten Zustände
	wieder hergestellt (=Restauration).
	\item Fürst Mitternich kommt an die Spitze von Österreich und führt die Zensur ein. Es dürfen keine politische Meinung gebildet werden, weder in der Kunst noch in Büchern oder im Theater.
	\item Die Kunst wird in das Privatleben verlagert und man schreibt über einfache Dinge im Leben (Arlalbert Stifter): "Einfache Dinge wie das Überkochen von Milch, ist genau so wichtig wie ein Vulkanausbruck (da es der gleiche Vorgang ist)." (Sanftes Gesetz, Bunte Steine)
	\item die Biedermeier hat auch Auswirkungen auf die Kleidung und Innenarchitektur. Nur die wohlhabenden Leute konnten sich Biedermeier Möbelstücke und Kleidung leisten.
	\item 1948 Revolution: diese wird aber blutig niedergeschlossen	
	
	\end{itemize}
	
\newpage	
	
\section{Junges Deutschland (1830 bis 1835)}

\subsection{Allgemeines}
Während der Julirevolution (1830) ergreift das Bürgertum die Macht in einem liberalem Königreich (Frankreich). Das Junge Deutschland (1830 bis 1835) ist der Name für eine literarische Bewegung junger, liberal gesinnter Dichter in der Zeit des Vormärzes. Sie wollen auch in der Politik mitmischen und schreiben über die Politik. Das Ziel dieser Gruppe von Dichtern war es, sich gegen die Politik Metternichs und der Fürsten des Deutschen Bundes. 1835 wurden Schriften auf Beschluss des Bundestages verboten.
\newline
\newline
Im Unterschied zu Zeitgenossen wie Georg Büchner oder zur späteren Dichtergeneration des Vormärz um Georg Herwegh, Ferdinand Freiligrath, Heinrich Heine und August Heinrich Hoffmann von Fallersleben ging es den Jungdeutschen allerdings nicht primär um einen politischen Umsturz. Sie strebten vielmehr eine vollständig neue, liberale Gesellschaft an, in der keine Autorität mehr ohne weiteres akzeptiert werden sollte. Für sie war Politik nur ein Bereich unter vielen, neben Moral, Religion, Ästhetik.
\newline
\newline
Georg Büchner und Heinrich Heine wurden verfolgt, da sie gegen das System Werke schrieben. Heinrich Heine flüchtete nach Paris. Georg Büchner flüchtete nach Zürich (Schweiz), dort erkrankte er an Typhus und starb am 2.Februar 1837.
\newline
\newline
Das Junge Deutschland wehrte sich im Allgemeinem mit der lyrischen Textform, aber auch mit der Dramatik (Woyzeck).

\newpage

\section{Märzrevolution (1948)}

\subsection{Allgemeines}
Die Ziele der Märzrevolution waren die Erneuerung der Literatur (z.B. im dramatischen Bereich: Woyzeck), das Recht auf Bildung für Frauen, Zensur abschaffen, gegen die Willkür der absoluten Herrscher zu arbeiten, gegen Kleinstaaten und die Einführung einer demokratischen Verfassung.
\newline
\newline
\subsection{Wichtige Autoren}

	\begin{itemize}
	\item Heinrich Heine (Literatur: Deutschland - Ein Wintermärchen), Christian Dietrich Grabbe (Satire oder Zeitkritik im Vordergrund)
	\item Ludwig Feuerbach - Religion ist Opiun für das Volk (nur dazu da, um das Volk zu beruhigen)
	\item Georg Büchner eher sozialistische Kampfschriften
	\end{itemize}\

\newpage

\section{Bürgerlicher oder Poetischer Realismus (1848 bis 1885)}

\subsection{Allgemeines}
Was tut sich historisch in der Zeit? - 1848 findet die bürgerliche Revolution statt. Das Militär verhilft den Kaiser zum Sieg, indem sie die Aufstände blutig niederschlagen. Franz Joseph dankt das Militär für ihre Unterstützung.
\newline
\newline
Realismus beschäftigt sich nicht mit der Politik sondern um allgemein menschlichen Problemen. Menschen, die im Mittelpunkt stehen, sind einfache Leute (z.B. Bauern, ...). Im Realismus werden Dinge so beschrieben wie sie sind und nicht wie man sie gerne hätte. Im Unterschied dazu steht die Klassik, die die idealen Eigenschaften beschreiben. Novelle ist zu dieser Zeit eine sehr beliebte Textform.
\newline
\newline
Im Jahre 1871 kommt es zur Einigung der Fürstentümer von Preußen und es entsteht ein einheitliche Reich namens Deutschland. Der Imperialismus (streben nach Weltherrschaft) ist eine Erscheinung dieser Zeit. Staaten wollen weltweit Kolonien haben (billige Rohstoffe). Die Deutschen wollen dort mitmischen $\rightarrow$ Aufstockung der Flotte in den Kolonien.
\newline
\newline
Die Werke des poetischen Realismus stellen eine Soseins-Dichtung und nicht eine Sollseins-Dichtung. Indem der Mensch geschildert wird wie er ist und nicht wie er sein soll (großer Unterschied zur Klassik). Vorwiegend werden zwischenmenschliche Probleme aufgezeigt und beschrieben (häufig Probleme von einfachen Leuten).
\newline
\newline
Zu Beginn lehnte sich der Realismus an die Philosophie von Ludwig Feuerbach an. Späterer Vertreter des Realismus waren hingegen von einem starken Pessimismus beeinflusst.
\newline
\newline
Ismen:
\begin{itemize}
\item Nationalismus
\item Liberalismus
\item Antisemitismus
\item Industrialisierung 
\end{itemize}

\subsection{Künstler}
\begin{itemize}
\item Gottfried Keller (1819-1890
	\begin{itemize}
	\item ist auf dem Weg ein Maler zu werden, doch leider nicht sehr erfolgreich. Daher kehrt er  nach Hause zurück (Schweiz). ''Der grüne Heinrich'' ist eine Biografie über Gottfried Keller. Gottfried Keller hat eine bekannte Novellensammulung geschrieben - ''die Leute von Seldwyla''.
	\end{itemize}
\end{itemize}

\subsection{Julia und Romeo am Lande}

Es geht um die Bauernfamile (Mans und Marti). Bauern halten isch für rechschaffente leute, die spezielle prinzipien haben und für Bodenbehaftung, Einfacher Umgang usw. ABER sie haben ein Problem. BEide besitzen einen Acker der an einem dritten Acker angrenzt. Schwarzer Geiger ist nicht gesellschaftsfähig - man nimm sie nicht war. Er Besitz den dritten ACker und die beiden anderen Bauern pflügen  jedes jarh imme mehr von dem Schwarzen Geiger zu ihrem grundstück dazu. Mans -> Sali; Marti -> Vrenchen.
\newline
\newline
Das Geschehen steigt ein als die beiden Bauern plüfen und ihre kindern spielen auf den Acker des Schwarzen Geigers. Ortsmotiv (kommt öfters vor). Der ORt wird sehr genau beschrieben (realistisch). Am ende des pfügens nimmt jeder bauer einen Furche des Schwarzen Geigers. 
Novelle DREIGETEILT. Erste Teil zu Ende. Gemeinsam nimmt man em SChwarzen Geiger langsam den Grund weg.
\newline
\newline
Zweiter Abschnitt: der Acker des schwarzen Geigers wird versteigert. Die Beiden Bauern steigert mit. Mans kauft sich das Grundstück. Er weiß, das Marti immer ein Stück von jetzt seinem Acker hat. Gericht schalten sich ein. Gerichte sind nicht am Land sondern in der Stadt und sind teuer. Besitz der Bauern wird immer weniger, davon Profitieren die Anwälte. Besitz wird so wenig das der MAns sein Grund verkaufen muss und zieht in die Stadt. Die Jugendlichen verlieren den Kontakt. Armut bricht aus. Und in dieser Zeit kommt es zum Höhepunkt. Sie treffen sich mit den Jugendlichen. Beiden Väter mit Kinder treffen sich. Steg über den Fluss. Die Väter schlagen sich gegenseiteig zusammen. Die Jugendlichen verlieben sich miteinander. Naturerscheinung stärkt ihre Liebe. Sie sind unzertänlich. Sie vereinmaren ein Treffen auf dem Acker vom Schwarzen Geiger.  Bei diesem Treffen passiert die Katastrophe. Sali schlägt MArli nieder das er ins Krankenhaus kommt. Sie haben keine Zukunft mehr. Dritter Akt: Beide ziehen sich gemeinsam am Wochenende durch das Volksfest. Sali kauft ein Herz ... beide treffen den schwarzen geiger, sie haben die Möglichkeit mit ihm umher zu wander. Beide Jugendlichen verbringen ihre Hochzeitsnacht auf einen Floss und ertrinken. Schlechte Moral der Jugendlichen in der Zeitung. Sprache sehr realistisch (blumige). Verwendet lange Sätze, atribute (beschreibung der Leute, Acker, sehr realistisch). Figuren sind sehr einfache Leute. Nichts wird beschönigt.

\subsection{Autoren zu dieser Zeit}
\begin{itemize}
\item Theodor Storm (1817-1888)
	\begin{itemize}
	\item Schimmelreiter (Norden Deutschlands)
	\end{itemize}
\item Theodor Fontane (1819-1898)
	\begin{itemize}
	\item Effi Briest (Ehebruch der Frau)
	\end{itemize}
\item Marie von Ebner-Eschenbach
\end{itemize}

\subsection{Wichtige Philosophen}

\begin{itemize}
\item Ludwig Feuerbach - alles Transzendente existiert für ihn nicht (Religionen ist Opium für das Volk). Nur fassbare Dinge gibt es, Religion gibt es nicht und ist nur dazu das Volk zu beruhigen und zu stärken.
\item Arthur Schogenhauer - Zitat: ''Die Welt ist die Äußerung einer unvernünftigen und blinden Kraft; in ihr zu leben heißt Leiden. ''. 
\end{itemize}	

Wichtige Ereignisse:

\begin{itemize}
\item 1866 - Krieg zwischen Preußen und Österreich $\rightarrow$ Österreicher verlor den Krieg (Hauptgrund ihres Versagens waren ihrer Meinung nach die veraltete Technologie)
\item 1871 - wird der preußische König Wilhelm der Erste zum deutschen Kaiser ernannt. Der preußische Ministerpräsident Otto von Bismarck wurde deutscher Reichskanzler. Damit wurde Preußen Teil eines neu gegründeten Deutschen Reiches.
\end{itemize}

\newpage

\section{Naturalismus (1880 bis 1900)}

\subsection{Allgemeines}

Der Begriff Naturalismus kommt vom lateinischen Wort "natura" (Natur) - auf Naturbeobachtung beruhend. Naturalismus allgemein bezeichnet eine Stilrichtung, bei der die Wirklichkeit ohne jegliche Ausschmückungen oder subjektive Ansichten exakt abgebildet wird. Der Naturalismus gilt auch als Radikalisierung des Realismus. Es sind alle literarischen Strömungen (Lyrik, Epik, Dramatik) vertreten. 
\newline
\newline
Keine Verantwortung für die Dinge, die man tut. Man hält sich nicht an seine eigene Moral. Die Menschheit ist verlogen und verbreitet Lügen und ist von Natur aus schlecht. Der Mensch ist nicht mehr für das Verantwortlich was er tut, weil er von anderen Faktoren (Erbgut und Milieu) eingeschränkt ist, dass er keinen freien Entscheidungsmöglichkeit mehr hat
\newline
\newline
Der Naturalismus beruhte nicht allein auf den Erkenntnissen der Naturwissenschaften, z. B. Charles Darwins Evolutionstheorien, er wurde auch stark von der Philosophie des Positivismus beeinflusst. Ist im gesamt europäischen Raum vertreten, Schwerpunkt in Russland, Frankreich und Skandinavien.
	
\subsubsection{Historische Hintergründe}	

Zu Beginn der 1880er Jahre kam es zu großen Fortschritten und Weiterentwicklungen in den Wissenschaften, z. B. 1884 wurde die Dampfturbine, 1887 die Schallplatte und 1893 der Dieselmotor erfunden.

\subsubsection{Politische Hintergründe}

Die Einigung Preußens sorgt für Unruhe im Europäischen Raum, weil eine neues Land in den Imperialismus folgen. Nationalismus ist stark vertreten. Der Antisemitismus ist deutlich spürbar (Juden-Hass). Rassismus ist in gewissen Maßen auch vertreten. Der Zustand des Proletariat ist katastrophal, Arbeiter werden ausgebeutet, keine gesetzliche Regeln für das Arbeiten. Der Absolutismus wird gehalten. Die Staaten teilen sich in Blöcken ein Ontoe (Frankreich, England, Serbien, ...).	
\newline
Bestimmend für die innen- und außenpolitische Entwicklung war Reichskanzler Bismarck. Im Deutschen Reich und in Europa wurde durch ihn eine gewisse Stabilität geschaffen, die erst wieder abnahm, als Bismarck 1890, wegen politischen Differenzen mit dem neuen Kaiser Willhelm II., zurücktreten musste.
	
\subsubsection{Literatur}

Zum ersten Mal werde in der Literatur hässliche, krampfhafte, seltsame Dinge nicht ausgeklammert. Schwerpunkt ist das Drama und der Roman. Es wird über alles Geschrieben auch über  abstoßende Dinge, die Soziale Frage wird angeschnitten und thematisiert (z.B. Totschlag, Inzest, Prostitution, wurde bis jetzt verschwiegen $\rightarrow$ Emile Zola - Nana). Die Wissenschaft finden einen Weg in die Literatur. Man baut auf wirtschaftliche Erkenntnisse auf (z.B. Evolutionstheorie, Mileutheroie, ...). Die Melieutheorie besagt, dass nicht nur die Gene den Charakter formen, sondern auch das Milieu in dem man sich aufhält. Transzendentes wird völlig ausgeschlossen (es gibt keine Gott, kein Leben nach dem Tod, ...). Der Dialekt findet Eingang in die Literatur (z.B. Vor Sonnenaufgang). Der Autor wird gefordert zu rescher schieren, um keine Ungenauigkeiten und falsche Fakten in das literarische Werk zu bringen.
\newline
Beim naturalistische Drama gibt genau vor wie die Bühne aussehen muss, wie die Figuren auszusehen haben und so weiter. 
Merkamale des naturalistischen Dramas (z.B. Vor Sonnenaufgang):

\begin{itemize}
\item Figuren haben keine Fallhöhe
\item Dialekt
\item Vererbungslehre, Milieutheorie, Soziale Frage wird angesprochen
\item es gibt keinen Ausweg (z.B. Freitod)
\item das Hässliche wird nicht ausgespart
\end{itemize}

\subsubsection{Deutschsprachige Autoren}

\begin{itemize}
\item Gerhard Hoffmann - Vor Sonnenaufgang
\item Arno Holz (eher in Vergessenheit geraten)
\end{itemize}

\subsubsection{Andere Autoren}

\begin{itemize}
\item Leo Tolstoi (Russe)

	\begin{itemize}
	\item Krieg und Frieden (geht um die Figur Napoléons)
	\item Anna Karenina (Ehebruch der Frau, Endet damit das die Frau sich vor den Zug wirft und stirb)
	\end{itemize}
	
\item Fjodor Dostojenoski (Russe) - wird gefangen und zum Tode verurteilt wird aber in der letzten Sekunde begnadigt und landet in ein sibirischen Arbeitslager.

	\begin{itemize}
	\item Schuld und Sühne (Frage ob mein Morden darf)
	\item Brüder Karamaow 
	\item Aufzeichnungen aus seinem Totenhaus
	\item Der Spieler
	\end{itemize}
	
\item Emile Zola (Franzose)
	
	\begin{itemize}
	\item Germinal (Arbeitslager; Dreifuß)
	\item Nana (Prostituierte, sie landet in der Gosse und wird von "Jack the Ripper" ermordet).
	\end{itemize}
	
\item Gustave Flaubert (Franzose)

	\begin{itemize}
	\item Madam Bovary (Ehebruch der Frau, Frau ersticht den Mann, da er sie in der Öffentlichkeit bloßstellt)
	\end{itemize}
	
\item Henrik Ibsen (Norwegen)

	\begin{itemize}
	\item Die Wildente (geht es um die Lebenslüge, die Erwachsenen schwindeln Werte und Gefühle vor)
	\item Gespenster (Lebenslügen)
	\end{itemize}		
	
\end{itemize}

\newpage

\section{Expressionismus}


\section{Interpretationsmethoden}



\end{document}

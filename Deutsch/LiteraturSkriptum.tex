\documentclass{report}

\usepackage[utf8]{inputenc}
\usepackage{dsfont}
\usepackage[german]{babel}

\title{Literatur}
\author{Alin Porcic}

\begin{document}

\maketitle
\newpage
\tableofcontents
\newpage

\chapter{Literarische Gattungen}
\subsection{Lyrik}

Mit der Lyrik ist die liedhafte Dichtung gemeint, die meistens strophenförmig aufgebaut ist. Sie kann auf zwei verschiedene
Möglichkeiten gereimt sein:
	\begin{itemize}
	\item Stabreim: gleicher Anlaut z.B. Mann und Maus
	\item Endreim: Gleichklang ab der letzen betonten Silbe
	\end{itemize}

\subsection{Dramatik}

Die Dramatik ist eine sehr alte literarische Gattung, die speziell für die Bühne gedacht ist. Die beschriebenen Ereignisse spielen 
in der Gegenwart und der Inhalt kann im Monolog und Dialog wiedergegeben werden. Das Drama wird durch Bühnenbild, Akustik, Gestik, Mimik etc. verstärkt.

Die Regeln des Aristoteles:

	\begin{itemize}
	\item Große Fallhöhe der Figuren
	\item Drei Einheiten müssen gewahrt werden
		\begin{itemize}
		\item Einheit des Ortes (kein Ortswechsel)
		\item Einheit der Zeit (das Stück darf nich länger als einen Tag dauern)
		\item Einheit der Handlung (es darf nicht mehrere Handlungsstänge geben)
		\end{itemize}
	\item Kataris (die Zuschauer dazu bringen, nicht die gleichen Fehler wie im Stück zu machen)
	\item Hochsprache muss verwendet werden
	\end{itemize}

Das Drama hat zwei Formen:

	\begin{itemize}
	\item Geschlossene Drama
		\begin{itemize}
		\item wenig Figuren
		\item Handlung auf das Ende hin einstränig
		\item fester Schluss
		\end{itemize}
	\item Offenes Drama
		\begin{itemize}
		\item viele Figuren
		\item mehrere Handlungsstänge
		\item offenes Ende
		\end{itemize}
	\end{itemize}
	
Das klassische Drama von Aristoteles hat fünf Akten:

	\begin{itemize}
	\item 1.Akt (Einleitung)
	\item 2.Akt (Steigerung der Verwicklung)
	\item 3.Akt (Höhepunkt)
	\item 4.Akt (Umschlag und fallende Handlung)
	\item 5.Akt (Katastrophe, Rettung, Lösung; ist abhänig vom Dramentyp)
	\end{itemize}
	
Dramentypen:

	\begin{itemize}
	\item Tragödie: Endet mit dem Untergang
	\item Komödie: Löst die Verwicklung unter Belustigung auf
	\item Schauspiel: Held geht nicht unter / Mittellage
	\end{itemize}

\subsection{Epik}

Die Epik ist die erzählende Dichtung. Es taucht ein Erzähler auf, der die Geschichte vorträgt und dabei kann er verschiedene
Erzählverhalten aufweisen:

	\begin{itemize}
	\item auktoriales Erzählverhalten: Der Erzähler hat die olympische (göttliche) Position und kann gelegentlich in der Ich/Er-Form kommentierend eingreifen. Zudem weiß er die Zukunft und das Ende der Handlung.
	\item neutrales Erzählverhalten: Der Erzähler ist ein Feldherr, er hat einen guten Überblick, jedoch schlechte Detailkenntnisse über die Lage.
	\item persönliches Erzählverhalten: Der Erzähler ist selbst am Geschehen  beteiligt, weiß die Gefühle seiner Figur und hat im Gegensatz zum neutralem Erzählverhalten bessere Detailkenntnisse.
	\end{itemize}
	
Der Erzähler kann verschiedene Erzählhaltungen haben:

	\begin{itemize}
	\item kritisch
	\item ironisch
	\item neutral
	\item ...
	\end{itemize}
	
Epische Texte sind gekennzeichnet durch:

	\begin{itemize}
	\item Inhalt: äußeres Gerüst einer Geschichte (Handlungsverlauf und Figurenkonstalation; Inhaltsangabe)
	\item Fabel: Inhalt auf das Äußerste reduziert (wenig Sätze)
	\item Stoff: Erlebnis, Ergebnis die dem Autor zum Schreiben angeregt haben
	\item Stillage: Art und Weise der sprachlichen Darstellung 
	\end{itemize}	
	
Epische Textsorten:

	\begin{itemize}
	\item Roman: Es gibt kein Merkmal, dass für alle Romane zutrifft, aber meistens erzählen Romane eine mehrsträngige Geschichte über
	einen längeren Zeitraum mit vielen Figuren. Der Begriff Roman heißt übersetzt romanisch und bedeutet Volkssprache. Es gibt viele verschiedene
	Romane z.B. Liebesroman, Thrillerroman, Science-Fiction-Roman, Abenteuerroman usw.
	\item Erzählung: Die Erzählung beschränkt sich auf einen engeren Weltausschnitt. Sie hat wenig Figuren, jedoch sind die erzählten Ereignisse in
	der realen Welt umsetzbar.
	\item Kurzgeschichte: Kurzgeschichten haben eine geringe Zeitspanne, die Figuren werden nicht entwickelt, sondern kommen in einem krisenhaftem
	Moment vor und das prägt ihr auftreten. Ihr Charakter ergibt sich aus ihrem handeln und ihren aussagen. Hierbei wird einfache knappe Sprache 
	verwendet und der Schluss ist meistens offen.
	\item Novelle: Die Novelle kommt aus dem Italienischen und heißt Neuigkeit (Zeit: Renaissance). Sie handeln von besonderen Ereignissen, die nicht
	alle Tage vorfallen. Aufbau einer Novelle:
		\begin{itemize}
		\item Rahmenerzählung = nennt den Erzählanlass
		\item Binnengeschichte = erzählt die eigentliche Geschichte
		\end{itemize}
	\item Epos: Das Epos kommt aus dem Griechischen und wird meistens in Fersen kunstvoll gestaltet. Sie handeln meistens um dramatische 
	Ereignisse z.B. Kriege, Leidenschaften. Einer der bekanntesten Dichter ist Homer mit der Ilias und Odysee.
	\end{itemize}
	
\section{Biedermeier 1915 bis 1948 (=Vormärz)}

Wichtiger Ereignisse:

	\begin{itemize}
	\item 1915 wird in Waterloo besiegt und auf die Insel St.Helena verbannt
	\item 18.September.1915 findet der Wiener Konkress statt. Dort werden neue Grenzen und Regeln festgelegt und es werden die alten Zustände
	wieder hergestellt (=Restauration).
	\item Fürst Mitternich kommt an die Spitze von Österreich und führt die Zensur ein. Es dürfen keine politische Meinung gebildet werden, weder in der
	Kunst noch in Büchern oder im Theater.
	\item Die Kunst wird in das Privatleben verlagert und man schreibt über einfache Dinge im Leben (Arlalbert Stifter): "Einfache Dinge wie das Überkochen 
	von Milch, ist genau so wichtig wie ein Vulkanausbruck (da es der gleiche Vorgang ist)." (Sanftes Gesetz, Bunte Steine)

	\item 1948 Revolution: diese wird aber blutig niedergeschlossen	
	
	\end{itemize}
	
\chapter{Deutschland}

\chapter{Märzrevolution}



Ziele der Märzrevolution waren:

	\begin{itemize}
	\item Erneuerung der Literatur (z.B. im dramatischen Bereich: Woyzeck)
	\item Recht auf Bildung für Frauen
	\item gegen die Zensur schreiben und für die Einführung der Pressefreiheit
	\item gegen die Willkür der absoluten Herrscher und für das Recht auf Freiheit, Gleichheit der Bürger
	\item gegen die Kleinstaaten
	\item für eine demokratische Verfassung (Trennung von Staat und Amtskirche)
	\end{itemize}

Wichtige Autoren:

	\begin{itemize}
	\item Heinrich Heine (Literatur: Deutschland - Ein Wintermärchen), Christian Dietrich Grabbe (Satire oder Zeitkritik im Vordergrund)
	\item Ludwig Feuerbach - Religion ist Opiun für das Volk (nur dazu da, um das Volk zu beruhigen)
	\item Georg Büchner eher sozialistische Kampfschriften
	\end{itemize}
	
Woyzeck:
1836 wird Woyzeck hingerichtet. Hinterlasst ein Buch als Fragment.

	\begin{itemize}
	\item nicht in akten, lose Scenenfolge
	\item die Hauptfigur hat keine Fallhöhe (Soldat) -> als Experiment verwendet
	\item lediges Kind mit Marie, bedrügt ihn
	\item Figurensprache: Soldat kann nicht sprechen nur stottern; Marie spricht aus der Bible
	\item Drama ist offen. Ausgang für Woyzeck ungewiss
	\item dieses Drama kann aus vielen Sichtweisen intepretiert werden (Interpretationsmethoden)
	\end{itemize}

	
\end{document}

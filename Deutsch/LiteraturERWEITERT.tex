\documentclass[a4paper]{report}
\usepackage[utf8]{inputenc}
\usepackage[german]{babel}

\begin{document}

\chapter{Bürgerlicher oder Poetischer Realismus (1848 bis 1885)}

Was tut sich historisch in der Zeit? - 1848 findet die bürgerliche Revolution statt. Das Militär verhilft den Kaiser zum Sieg, indem sie die Aufstände blutig niederschlagen. Franz Joseph dankt das Militär für ihre Unterstützung. 
\newline
\newline
Österreich-Preußischer Krieg - Österreich hat den Krieg verloren, weil sie schlechter Ausgerüstet waren als die Preußen.
\newline
\newline
1870/71 Frankreich-Deutscher Krieg.
1871 wird das deutschen Reiche gegründet.
Imperialismus (streben nach Weltherrschaft) ist eine Erscheinung dieser Zeit. Staaten wollen weltweit Kolonien haben (billige Rohstoffe). Deutschen und Preußische wollen dort mitmischen $\rightarrow$ Aufstockung der Flotte.
\newline
\newline
Realismus beschäftigt sich nicht mit der Politik sondern um allgemein menschliche Probleme. Menschen, die im Mittelpunkt stehen, sind einfache Leute (z.B. Bauern, ...).
\newline
\newline
Ismen:
\begin{itemize}
\item Nationalismus
\item Liberalismus
\item Antisemitismus
\item Industrialisierung
\end{itemize}

Joseph hält das Reich bis zu seinen Tod zusammen. Die Teilgebiete wollen unabhängig sein und Österreich will zu Deutschland, was der Adolf Hitler verwirklicht.
\newline
\newline
Wichtige Philosophen:
\begin{itemize}
\item Ludgwig Feuerbach - alles Transzendente existiert für ihn nicht (Religionen ist Opium für das Volk)
\item Arthur Schogenhauer - Zitat: ''Die Welt ist Äußerung einer unvernünftigen und  blinden Kraft. In ihr zu leben heißt leiden.'' Kulturpessimismus.
\end{itemize}

Die Werke des poetischen Realismus stellen eine Soseins-Dichtung und nicht eine Sollseins-Dichtung. Indem der Mensch geschildert wird wie er ist und nicht wie er sein soll (großer Unterschied zur Klassik). Vorwiegend werden zwischenmenschliche Probleme aufgezeigt und beschrieben (häufig Probleme von einfachen Leuten).
\newline
\newline
Gottfried Keller 1819-1890 - ist auf dem Weg ein Maler zu werden, doch leider nicht sehr erfolgreich. Daher kehrt er  nach Hause zurück (Schweiz). ''Der grüne Heinrich'' ist eine Biografie über Gottfried Keller. Gottfried Keller hat eine bekannte Novellensammulung geschrieben - ''die Leute von Seldwyla''.
\newline
\newline
Novelle ist zu dieser Zeit eine sehr beliebte Textform.
\newline
\newline
Autoren zu dieser Zeit:
\begin{itemize}
\item Theodor Storm (1817-1888)
\item Theodor Fontane (1819-1898)
	\begin{itemize}
	\item Effi Briest (Ehebruch der Frau)
	\end{itemize}
\item Marie von Ebner-Eschenbach
\end{itemize}

\end{document}
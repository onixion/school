\documentclass[a4paper]{article}
\usepackage[utf8]{inputenc}
\usepackage[german]{babel}

\title{Prüfungen}
\author{Alin Porcic}

\begin{document}

\maketitle
\newpage

\section{26.05.14}

\subsection{Zens}

\subsubsection{Naturalismus}

Stark beeinflusst von Naturwissenschaften. Alle literarische Gattungen sind vertreten. Schwerpunkt im Naturalismus ist die Dramatik, Epik. Wo kommen epische Werke vor: ?. Es wird alles hässliche,krampfhafte und abstoßende Dinge  beschrieben. Welche wissenschaftliche Erkenntnisse werden in den Werken oftmals verwendet: Evolutionslehre (Vererbungslehre);
Philosophie: Positivismus, Alles Zanstredente gibt es nicht (Feuerbach)
\newline
Werke mit Namen: -

\subsubsection{Sprachbuch: Gedicht Interpretation; Goethe: Willkommen und Abschied}

Interpretationsmethoden: Werkimanent (durchlesen); Reim: Kreuzreim (ababcdcd); Rhythmus: Jambus, Jamben; Natur ist wichtig (=Spiegel der Seele), wie wird sie dargestellt? - Feindlich, als Bedrohung, -

\subsection{Egrezberger}

\subsubsection{Naturalismus}

Autoren + Werke
\newline
Gerhard Hauptmann - Vor Sonnenaufgang
\newline
Merkmale zeigt dieses Drama auf:
-
\newline
Andere Autoren: -

\subsection{Jurily}

\subsubsection{Naturalismus}

Verbreitung: ganz Europa
Schwerpunkte: Russland, Skandinavien, Frankreich
Autoren: 
\begin{itemize}
\item Fjodor Doschewski: wird kurz bevor er hingerichtet wird begnadigt; welches schreibt er ''Auszeichnungen aus seinem Totenhaus''. 
\item Leo Tolstoi: -
\end{itemize}

\section{06.03.14}

\subsection{SToj}

\subsubsection{Gedichtinterpretationen}

Werkimanent $\rightarrow$ nur text, schaut auf style, Text
Sprachanalytische $\rightarrow$ Wortgut, stilistische Figuren, Dialekt,
Bigroafische $\rightarrow$ wer war der Autor, wo hat er gelebt, bei welchen Texten sinnvoll / hat nur Sinn wenn: Wenn die Geschichte mit seinem Leben was zu tun hat (z.B. Schöne Tage, verwendet nur einen anderen Namen)
Phsychologische $\rightarrow$ spielt dann eine Rolle wenn die Hauptpersonen psychischen Störungen aufweisen (z.B. Der Vorleser)
Soziologische $\rightarrow$ Umfeld, Politiker, Historische Geschichte (z.B. Winterbucht, normal das man Gewalt anwendet, ... z.B. Kameramörder, Massenmedien)
\newline
Lyrischen Texten $\rightarrow$ Reime, Stabreim (=Alliteration)(Gleicher Anlaute), Endreim (Gleichklang ab der letzten Betonten Silbe)
Alchemismus (=Beschönigung), Ellipse (=Verkürzung; Ende Gut, alles Gut), 

\subsection{Jurily}

\subsubsection{Wojzeck}

Autor: Georg Büchner 
Welcher Epoche zuzuordnen: Junges Deutschland (Zeit: 1830 bis 1835); Biedermeier (1815 bis 1848); 1815 Wiener Kongress; 1848 (Märzrevolution)


\subsection{Dani}

1848 Märzrevolution; Bürger gegen den Absolutismus (=Adelige, hat die politische Stellung ; Fürst Metternich); 
Nach 1848 spricht man von Aufgeklärter Absolutismus (weil die Aufklärung); Franz Joseph (über 60 Jahre regiert er; bis 1916) übernimmt Österreich.

Junges Deutschland (1830 bis 35) $\rightarrow$ Büchner wird verfolgt
Welche Gattungen: Lyrik und Dramatik; speziell Lyrik, weniger Dramatik

Naturalismus $\rightarrow$ Ismen; 
Kommunismus (Marx und Engels), wird verboten.

Naturalimus $\rightarrow$ Erfindungen (Flugzeuge, Schallplatte, Motoren, ...)

Realismus Literatur : niedrige Fallhöhe, sehr genaue Beschreibung des Ortes, Zwischenmenschliche Konflikte zwischen einfachen Leute (Zillertaler)

\subsection{Hell}

Wiener Volkstheater Merkmale: niedriger Fallhöhe, keine Hochsprache, vorwiegend Komödien (Happy End), Vorbild für Zukünftige Theaterstücke
Volkstheater und Volksstück unterscheide? - Volksstück kein Happy End, ernsthafte Probleme, Ausgrenzungen, menschlichen Schwächen; ähnliche oder gleich: Fallhöhe (es geht um die gleiche Leute; Landleute); erstes Volksstück: ''Ödön von Horvath" - "Mit den Geschichten aus den Wienerwald'', Felix Mitterer, Franz Gröz, zwischenmenschliche Probleme aufzeigen
\newline
Realismus: geht es auch um zwischen menschliche Probleme, Novelle (z.B. Theodor Storm ''Schimmelreiter''(Dämme Nord Deutschland), Gottfried Keller ''die Leute von Selvüla'';) 
\newline
Novelle (Rennasance)  muss ein spezielles Ereigniss sein , ein Handlungsstrang, pornografische (pokatscho)

\end{document}
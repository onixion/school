\documentclass[a4paper]{article}
\usepackage[utf8]{inputenc}
\usepackage[german]{babel}

\title{Prüfungen}
\author{Alin Porcic}

\begin{document}

\maketitle
\newpage

\section{26.05.14}

\subsection{Zens}

\subsubsection{Naturalismus}

Stark beeinflusst von Naturwissenschaften. Alle literarische Gattungen sind vertreten. Schwerpunkt im Naturalismus ist die Dramatik, Epik. Wo kommen epische Werke vor: ?. Es wird alles hässliche,krampfhafte und abstoßende Dinge  beschrieben. Welche wissenschaftliche Erkenntnisse werden in den Werken oftmals verwendet: Evolutionslehre (Vererbungslehre);
Philosophie: Positivismus, Alles Zanstredente gibt es nicht (Feuerbach)
\newline
Werke mit Namen: -

\subsubsection{Sprachbuch: Gedicht Interpretation; Goethe: Willkommen und Abschied}

Interpretationsmethoden: Werkimanent (durchlesen); Reim: Kreuzreim (ababcdcd); Rhythmus: Jambus, Jamben; Natur ist wichtig (=Spiegel der Seele), wie wird sie dargestellt? - Feindlich, als Bedrohung, -

\subsection{Egrezberger}

\subsubsection{Naturalismus}

Autoren + Werke
\newline
Gerhard Hauptmann - Vor Sonnenaufgang
\newline
Merkmale zeigt dieses Drama auf:
-
\newline
Andere Autoren: -

\subsection{Jurily}

\subsubsection{Naturalismus}

Verbreitung: ganz Europa
Schwerpunkte: Russland, Skandinavien, Frankreich
Autoren: 
\begin{itemize}
\item Fjodor Doschewski: wird kurz bevor er hingerichtet wird begnadigt; welches schreibt er ''Auszeichnungen aus seinem Totenhaus''. 
\item Leo Tolstoi: -
\end{itemize}



\end{document}
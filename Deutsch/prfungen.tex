\documentclass[a4paper]{article}
\usepackage[utf8]{inputenc}
\usepackage[german]{babel}

\title{Prüfungen}
\author{Alin Porcic}

\begin{document}

\maketitle
\newpage

\section{26.05.14}

\subsection{Zens}

\subsubsection{Naturalismus}

Stark beeinflusst von Naturwissenschaften. Alle literarische Gattungen sind vertreten. Schwerpunkt im Naturalismus ist die Dramatik, Epik. Wo kommen epische Werke vor: ?. Es wird alles hässliche,krampfhafte und abstoßende Dinge  beschrieben. Welche wissenschaftliche Erkenntnisse werden in den Werken oftmals verwendet: Evolutionslehre (Vererbungslehre);
Philosophie: Positivismus, Alles Zanstredente gibt es nicht (Feuerbach)
\newline
Werke mit Namen: -

\subsubsection{Sprachbuch: Gedicht Interpretation; Goethe: Willkommen und Abschied}

Interpretationsmethoden: Werkimanent (durchlesen); Reim: Kreuzreim (ababcdcd); Rhythmus: Jambus, Jamben; Natur ist wichtig (=Spiegel der Seele), wie wird sie dargestellt? - Feindlich, als Bedrohung, -

\subsection{Egrezberger}

\subsubsection{Naturalismus}

Autoren + Werke
\newline
Gerhard Hauptmann - Vor Sonnenaufgang
\newline
Merkmale zeigt dieses Drama auf:
-
\newline
Andere Autoren: -

\subsection{Jurily}

\subsubsection{Naturalismus}

Verbreitung: ganz Europa
Schwerpunkte: Russland, Skandinavien, Frankreich
Autoren: 
\begin{itemize}
\item Fjodor Doschewski: wird kurz bevor er hingerichtet wird begnadigt; welches schreibt er ''Auszeichnungen aus seinem Totenhaus''. 
\item Leo Tolstoi: -
\end{itemize}

\section{SToj}

\subsection{Gedichtinterpretationen}

Werkimanent $\rightarrow$ nur text, schaut auf style, Text\\
Sprachanalytische $\rightarrow$ Wortgut, stilistische Figuren, Dialekt,\\
Bigroafische $\rightarrow$ wer war der Autor, wo hat er gelebt, bei welchen Texten sinnvoll / hat nur Sinn wenn: Wenn die Geschichte mit seinem Leben was zu tun hat (z.B. Schöne Tage, verwendet nur einen anderen Namen)\\
Phsychologische $\rightarrow$ spielt dann eine Rolle wenn die Hauptpersonen psychischen Störungen aufweisen (z.B. Der Vorleser)\\
Soziologische $\rightarrow$ Umfeld, Politiker, Historische Geschichte (z.B. Winterbucht, normal das man Gewalt anwendet, ... z.B. Kameramörder, Massenmedien)
\newline
Lyrischen Texten $\rightarrow$ Reime, Stabreim (=Alliteration)(Gleicher Anlaute), Endreim (Gleichklang ab der letzten Betonten Silbe)\\
Alchemismus (=Beschönigung), Ellipse (=Verkürzung; Ende Gut, alles Gut), \\

\section{Jurily}

\subsection{Wojzeck}

Autor: Georg Büchner \\
Welcher Epoche zuzuordnen: Junges Deutschland (Zeit: 1830 bis 1835); Biedermeier (1815 bis 1848); 1815 Wiener Kongress; 1848 (Märzrevolution)\\

\section{Dani}

1848 Märzrevolution; Bürger gegen den Absolutismus (=Adelige, hat die politische Stellung ; Fürst Metternich);\\
Nach 1848 spricht man von Aufgeklärter Absolutismus (weil die Aufklärung); Franz Joseph (über 60 Jahre regiert er; bis 1916) übernimmt Österreich.\\

Junges Deutschland (1830 bis 35) $\rightarrow$ Büchner wird verfolgt\\
Welche Gattungen: Lyrik und Dramatik; speziell Lyrik, weniger Dramatik\\

Naturalismus $\rightarrow$ Ismen;\\
Kommunismus (Marx und Engels), wird verboten.\\
Naturalimus $\rightarrow$ Erfindungen (Flugzeuge, Schallplatte, Motoren, ...)\\
Realismus Literatur : niedrige Fallhöhe, sehr genaue Beschreibung des Ortes, Zwischenmenschliche Konflikte zwischen einfachen Leute (Zillertaler)

\section{Hell}

Wiener Volkstheater Merkmale: niedriger Fallhöhe, keine Hochsprache, vorwiegend Komödien (Happy End), Vorbild für Zukünftige Theaterstücke
Volkstheater und Volksstück unterscheide? - Volksstück kein Happy End, ernsthafte Probleme, Ausgrenzungen, menschlichen Schwächen; ähnliche oder gleich: Fallhöhe (es geht um die gleiche Leute; Landleute); erstes Volksstück: ''Ödön von Horvath" - "Mit den Geschichten aus den Wienerwald'', Felix Mitterer, Franz Gröz, zwischenmenschliche Probleme aufzeigen
\newline
Realismus: geht es auch um zwischen menschliche Probleme, Novelle (z.B. Theodor Storm ''Schimmelreiter''(Dämme Nord Deutschland), Gottfried Keller ''die Leute von Selvüla'';) 
\newline
Novelle (Rennasance)  muss ein spezielles Ereigniss sein , ein Handlungsstrang, pornografische (pokatscho)

\subsection{Frühlings}

Figuren\\ 
sind in der Pubertät, sie werden nicht aufgeklärt, sexuelle Regungen wiesen nicht was tun?\\
Melchior wird aufgeklärt und schreibt eine Inhaltsangabe für Moritz. Welche Folgen? Moritz bring sich um, das Schreiben von Melchior wird gefunden, er wird rausgeworfen und in die Anstalt geschickt.\\
Wendla wird schwanger, Eltern über Brief verständigt,\\
End-Scene -> am Grabstein von Wendla -> Geister (unecht), Moritz will Melchior zum Tod auffordern\\

Was ist Expressinismus? -> Gefühle wichtig, Allegorie (Stilmittel), Feindbild der Lehrer (die Generation der Väter), Sprach ist nicht expressinistisch oder Aufbau\\
Hauptthema: Selbstmord, Onanieren in der Gruppe (Onanieren), Homosexuelle (Ernst und Hänschen), Freunding von Moritz Isle ist ein bisschen Zweifelhaft, Wendla lässt sich schlagen vom Melchior, Lange nicht aufgeführt\\

Natrualismus / Expressinismsus :\\
Sprache (Marinetti): Neue Sprache erschaffen, Neue Verben schaffen,\\ 
Sprache Einfach im Naturalismus: Dialekt

\section{Marko}
Sturm und Drang\\
Junges Kraftgenie\\
Schiller: Räuber\\
Zwei Brüder: Karl Moore, Franz Moore (Betrüger)\\
Robin Hood motiv\\
Es gibt Todesopfer.\\
Vater verflucht Karl Moore\\
der Werther: Natur ist wichtig, Natur als Spiegel der Seele, Gefühlswert\\
Modern\\
Dramatische Formen\\
Dokuemtar Theater die Ermittelung\\
Wichtig: der Autor greif auf ein historisches Ereignis zurück, er muss rescherschien, Protokolle, Tonbandaufnahmen und Zeugen dienen als Rescherschematerial\\
die Aufgabe des Autors ist es "aufführbar" zu machen.\\
Jahrhundert 60er Jahre des 20Jahrhundert, 2 Weltkrieg\\
Die Ermittelung: handelt über ein Gerichtsprozess (Ausschwitz), wie lange dauert der prozess? - 2 Jahre\\
wie kürz der Autor den Inhalt? - er kürz die viele Zeugen auf 9 Personen

\section{Jurily 12.12}

Episches Drama - Bertholt Brecht\\
Sie wollen die Zuschauer mit dem Drama belehren.\\
Die Spannung wird herausgenommen, da sie den Zuschauer blendet.\\
Damit der Zuschauer Lehren aus dem Stück nehmen können, muss die Spannung herausgenommen werden.\\
V-Effekte: Lieder, Kommentare\\
Wieso V-Effekte: Auflösung der Spannung, der Zuschauer muss wiesen, dass er im Theater sitzt.\\
Mutter Courage vesteht nicht, dass man mit dem Krieg keine Geschäfte macht. In den Liedern findet Mutter Courage heraus, dass sie keinen Gewinn machen kann; Zwischenkriegszeit; Brecht möchte den Zuschauer vermitteln, dass man mit dem Krieg kein Geschäft macht.\\\\

Klassik - Viergestirn, Weimar\\
Warum Weimar ein guter Ort ist? - Goethe, Schiller und andere Autoren werden dort von den Fürsten dafür finanziert\\
Zeit: Ende 18Jahrhundert\\
Schiller stammt nicht aus Weimar, sondern Franfurt am Main\\

Merkmale der Klassik:\\
Objektivismus\\
Orientierung an der griechischen Antike: Was gefällt ihnen an der Kunst? - Estetisch, Perfekt, Ideale\\
Idealfigur ist die Iphigenie: die in sich ruhende Persönlichkeit\\
Sprache: gehoben, hochwertige Sprache\\
Gattungen: alle Gattung sind verteten, aber getrennt; Balladen sind lyrisch; Goethe hat viele Liebschaften gehabt\\
Scharlotte kommt nochmal vor: die Freundin des Werthers\\

\section{Stoj}

Romantik (1795-1835)
Aufgeteilt:
\begin{itemize}
\item Frühromantik
\item Hochromantik
\item Spätromantik
\end{itemize}

Der aus sich herauswachsende Person, nicht das allgemeingültige Einzelschicksal\\
Gattungen: Vorliebe für Vermischungen\\
Offener Aufbau\\
Beispiele für romanische Werk:\\
Fräulein von Skyderie:\\
Warum ist das ein romantisches Werk?\\
Kriminalroman / Novelle kann man sagen\\
Fräuline von Skyderie und der Kardinal -> sind besonder Figuren\\
Kardinal, was tut er? Goldschmied, er tötet seine Kunden, weil er sich nicht von seinen Produkten nicht trennen\\\\
Dokumentartheater (1960)\\
Beispiele:\\
Der Stellvertreter (Autor Rolf Ruth): Papst Pios, wieso der Papst nicht Stellung zum Holocaust genommen hat
Die Ermittlung (Peter Weiss): Frage? - In Frankfurt am Main finden der Prozess gegen die KZ-Leiter aus Ausschwitz (ist das größte KZ das jemals gebaut worden ist); liegt in Polen; Millionen Opfer; 11 Gesänge (=Oratorium); der Autor braucht zuverlässige Quellen; der Autor muss versuchen die Handlung aufführbar zu machen $\rightarrow$ er muss es kürzen; er fasst die Zeugen zu 9 Zeugen zusammen; die Zeugen bleiben anonym; Zyclon-B ist das Gas mit dem die Juden umgebracht wurden; 
Die Sache von Oppenhauer: ist der Oppenhauer noch für Amerika von nutzen, Kalter Krieg (Russland gegen Amerika); Wer hat als erstes die Wasserstoffbombe; die ersten sind die Amerikaner; Verdacht das der Oppenheimer mit den Russen zusammengearbeiten hat; Oppenheimer darf nicht mitarbeiten 

\section{23.01.2014: Martin}

\subsection{Beistrichsetzung}

Weil er keine Ahnung hatte, waren es zu wenige Punkte, um erfolgreich zu sein. (kausaler Gliedsatz), (HS), (erweiterter Infinitiv)\\
Karl Markus, der Erfinder des Benzinmotors, war berühmter als Otto Diesel, der Konstrukteur des Dieselmotors. (HS T1), (Apposition), (HS T2), (Apposition)\\
Wir treffen uns am Dienstag, den 23.1, um Fünfzehn Uhr, aber pünktlich. (HS), (Datum), (Zeitangabe), (entgegenstellende Konjuktion)\\
Zu siegen war immer seine Absicht, denn wer erfolgreich ist, der ist auch sonst mit dem Leben zufrieden. (HS), (HS), (Attributsatz) ('denn' leitet immer einen Hauptsatz ein!)

\subsection{Leutant Gustl}

Zeitlich wie zuzuordnen: Expressionismus, Arthus Schnitzler. Jahrhundert-Wende\\
Bedeutung für die Literatur: innere Monolog. Personales Erzählverhalten. innere Monolog ist eine spezielle Variante des personalen ERzählverhaltens (Gedanken,...)\\
Epische Textform: Novelle\\
Merkamle der Novelle: Renassians, Rahmen-Erzählung (erzählt den Erzählanlass), Binnengeschichte (ist das eigentliche Geschehen)\\
Rahmen-Erzählung: hat keine\\
Besondere Anlass: Bäcker trifft auf Leutant Gustl hinter der Oper in der Garderobe; kommt zum Streit, weil sich der Bäcker vordrängelt; Leutant Gustl kann sich nicht mit den Bäcker duellieren, da er ein Zivilist ist; Bäcker ist kräftig und Gustl hat keinen Chance.  

\section{23.01.2014: Thurner - Das Urteil}

Von Franz-Kafka; autobiografisches Werk; Kafka schreibt über seine Probleme; der Hass auf den Vater\\
Hauptfigur schreibt Briefe an einen Freund\\
sein VAter macht im Vorwürfe, weil die Hauptfigur seinem Freund nicht alles erzählt hat.\\
Vater und die Hauptfigur streiten und der Vater verurteilt ihn zum Tode

\section{23.01.2014: Widschwenter - Besuch der alten Dame}

Friedrich Dürer; Gattung: Dramatisch, Aufführung\\
Merkmale der Dramatik: für die Bühne gedacht\\
Wie ist das Drama aufgebaut?

Wie sind Dramen im Allgemeinen aufgebaut?\\
in Akten; 5 Atken nach Archistotelles\\
Wie ändert der erste Akt im Buch?\\
Als die Frau für den Kopf von Ill eine Millionen bietet. 'Ich bin verloren.' Ill erkennt, dass er nicht flüchten kann und sieht ein das er ein Verbrechen begannen hat.

\end{document}

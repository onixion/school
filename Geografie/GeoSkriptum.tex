\documentclass[a4paper]{report}
\usepackage{german}
\usepackage[utf8]{inputenc}

\title{Geofrafie Scriptum}
\author{Alin Porcic}

\begin{document}

\maketitle
\newpage

\tableofcontents
\newpage

\chapter{Märkte als Orte des Wirtschaftes}

Auf einem Markt treffen sich Anbieter und Konsumierer und deren unterschiedliche Interessen zeigen sich auch in der Preisbildung. Markttransparent bedeutet: vollständiger Überblick über Angebot bzw. Konkurenz ermöglicht gute Kaufentscheidung bzw. Preisbildungsentscheidung.

% HIER FEHLT NOCH ETWAS

	\begin{itemize}
	\item ''band wagon effect'': Der Effekt der Eintritt wenn ein Preis eines Produktes steigt, die Nachfrage aber nicht zurückgeht sondern vieleicht sogar
	nocht mehr steigt z.B. iPhone
	
	\item ''snob (value) effect'': Dieser Effekt betrifft Luxusgüter, die man einfach nur haben muss egal wie teuer.
	
	\item Nachfrageelastizität
		\begin{itemize}
		\item proportional: Preis steigt um 3\% - Nachfrage -3\%; Preis fällt um 2\% - Nachfrage +2\% (in der Theorie)
		\item elastisch: Preis steigt um 3\% - Nachfrage -5\%; Preis fällt um 2\% - Nachfrage +4\% (bei Produkten mit Alternativen; Luxusgüter, die nicht
		unbedingt nötig sind)
		\item unelastisch/starr: Preis steigt um 3\% - Nachfrage -1\% (bei lebensnotwendigen Produkten)
		\end{itemize}
	
	\item Kreuz-Preis-Elastizität:
		\begin{itemize}
		\item Substitiutionsgüter: z.B. Butter - Margarine
		
		Butterpreis steigt:
			\begin{itemize}
			\item private Auswirkung eher gering
			\item öffentliche Nachfrage geht zurück - Margerine wird eher gekauft
			\end{itemize}
			
		\item Komplementärgüter: z.B. Ski + Bindung, DvDs + Player, Auto + Reifen
			
		\end{itemize}
	\end{itemize}



Polypol  Olipol  Monopol


viele Anbieter; viele Nachfrager

wenige Anbieter 
viele Nachfrager
viele Anbieter
wenig Nachfrage

ein Anbieter
viele Nachfrager

\chapter{Ökonomisches Prinzip}
Sinnvoll und vernünftig Wirtschaften bedeutet nach dem Wirtschaftsprinzip handelen. Damit wird versucht viele Bedürfnisse trotz der begrenzten Mittel zu
befriedigen. Es gibt zwei unterschiedliche Wege:

	\begin{itemize}
	\item Minimalprinzip: Gegeben ist ein bestimmtes Ziel und gesucht ist der minimale Einsatz.
	\item Maximalprinzip: Gegeben ist ein bestimmter Einsatz und gesucht ist maximales Ziel.
	\end{itemize}
	
Öffentliche Stellen wie zum Beispiel Gemeinden vergeben die Aufträge nach Ausschreibungen, wobei normalerweise der Bestbieter zum Zug kommen.


\chapter{Produktionsfaktoren}

Die vier Produktionsgüter sind:

\begin{itemize}
\item Kapital
\item Arbeit 
\item Wissen
\item Grund und Boden
\end{itemize}

Beispiel Weizen:

Boden = Feld
Arbeit = Landwirt/-in
Kapital = Maschinen, Saatgut, ...
Wissen = Fachschule

\section{Grund und Boden (inklusive Rohstoffe)}

Benötigt man zum Anbau (Land und Forstwirtschaft), zum Abbau (Rohstoffe, Bergbau), zum Ausbau (Infrastruktur) und als Standort(Siedlungen, Betriebe).
Boden ist der einzige Produktionsfaktor der eindeutig knapp ist; nur in den seltensten Fällen lässt er sich vermehren (Neulandgewinnung in Japan, Dubai, Holland). Diese Knappheit führt zu steigenden Preisen. Durch unterschiedliche Bodenpreise kommt es zu unterschieldichen Nutzungen (z.B. in Zentren von Städten befinden sich mehr Geschäftsgebäude und Büros; höhere Bauten wegen weniger Platz). 

\section{Arbeit}
Die Höhe der Arbeitskosten beeinflusst den Einsatz den Produktionsfaktors Arbeit. Wenn die Löhne und die gehaltsabhänigen Abgaben zu hoch werden, beginnen Unternehmer Arbeit durch Kaptial zu ersetzen (Arbeiter mit Maschinen ersetzen). Das bedeutet das Arbeitskräft entlassen werden und Maschinen die Arbeit übernehmen (Vorteile: Arbeiten rund um die Uhr, weniger Störanfällig und genauer, keine Versicherung, keiner Gewerkschaft, usw.).\\
Einteilung der Arbeit:
\begin{itemize}
\item manuelle Arbeit
\item geistige Arbeit\\
\item Plfichtarbeit
	\begin{itemize}
	\item gesetzlich, tariflich z.B. Schule, Beruf
	\item moralisch z.B. Helfen im Haushalt, Pflege
	\end{itemize}
\item freiwillige Arbeit (z.B. Vereine, Hobby, Sport, ...)

\item selbstständig (z.B. frei Berufe, Künstler, Unternehmer, Hausarbeit, Hobby, Sport, ...)
\item unselbstständig (z.B. Arbeiter, Angestellte, Beamte, Schüler, ...)

\item Erwerbsarbeit (für jede Tätigkeit bei der man bezahlt wird; z.B. Arbeiter, Angestellte, Selbstständige, ...)
\item Nichterwerbsarbeit (z.B. Haushalt, ehrenamtliche Tätigkeit, Schularbeit, ...)
\end{itemize}

\subsection{Situation in Österreich}

Arbeitszeit(Stundenwoche):

\begin{tabular}{c|c}
\\1895 & 65
\\1913 & 57
\\1918 & 48
\\970 & 43 
\\1975 & 40
\\2011 & 40

\end{tabular}

Eine ständige Abnahme, der wöchentlicher Arbeitszeit, ist nur bei gleichzeitiger steigerung der Arbeitsproduktivität möglich
Unter Arbeitproduktivität versteht man das Verhältniss zwischen dem Produktionsergebnis und der Zahl der Beschäftigten oder der geleisteten Arbeitstunden.\\

Das Beispiel Österreich zeigt das Arbeit in den Industrieländern teuer geworden ist. Durch immer bessere Maschinen und eine bessere innerbetriebliche Organisation ist die Produktivität massiv angestiegen worden.\\

Berufe im Dienstleistungssektor dominieren heute, da in diesem Bereich Menschen nicht so leicht durch Maschinen ersetzt werden können.

\section{Kapital}

Kapital wird oft mit Geld gleich gesetzt. In der Wirstschaft versteht man jedoch unter Kapital alle Produktionsmittel (also Maschinen, Gebäude, Rohstoffe usw.), die zur Herstellung von Gütern oder zur Erbringung von Dienstleistungen dienen.

(ZETTEL)

Das Kapital kann aus Ersparnissen kommen oder geliehen werden. Man unterscheidet zwischen Eigen- oder Fremdkaptial.

Österreich:\\

Eigenkapitalquote 2009 in Prozent:\\

\begin{itemize}
\item Kleinstunternehmer (bis 10 Beschäftigten) 11,5\%
\item Kleinunternehmer (bis zu 50 Beschäftigten) 19,4\%
\item Mittelunternehmen (bis zu 250 Beschäftigten) 31,3\%
\item Großunternehmen (alles ab 250 Beschäftigten) 33,1\%
\end{itemize}

Eigenkapital dient in einem Unternehmen vor allem zur Deckung betrieblicher Risiken (Konjukturschwackung, Einführung neuer Produkte, Erschließung neuer Märkte, ...). Ohne Eigenkapital kann man kein Unternehmen gründen und man erhält auch kein Fremdkapital.

\section{Wissen}

Wissen ist heute für Unternehemen viel Wichtiger als Sachwerte wie z.B. Immobilien und Maschinen. Dieses intellektuelle Kapital ist setzt sich aus

\begin{itemize}
\item Wissen über Prozesse
\item Wissen über Produkte
\item Wissen über Kunden
\item Wissen, das Unternehmen erworben haben bzw. erwerben wollen
\item ''KnowHow''/Kompetenzen der Mitarbeiter/innen
\end{itemize}

Explizites Wissen: das Wissen das bereits exisiert bzw. jedem bewusst ist (mit diesem Wissen können anderen Menschen arbeiten und weiter verwenden).
\newline
\newline 
Implizites Wissen: das basiert auf eigenen Erfahrungen, Erinnerung und Überzeugung; dieses Wissen ist persölicher Besitz und macht den besonderen Wert des Trägers.
\newline
\newline
Die Statisiken zeigen, dass Bildung das wichtigeste Mittel ist, um am Arbeitsmarkt bestehen zu können. Niedrig qualifitierte Jobs werden durch den rasanten technologischen Fortschritt immer weniger bzw. die werden sie ausgelagert in die Schwellen und Industrieländern. Der Bedarf an hoch qualifizierten Arbeitskräfen wächst jedoch, vor allem in den sogenannten MINT-Berufen (Mathematik, Informatik, Naturwissenschaften, Technik). Zunehmend orientieren sich die Bildungsinhalte an den Bedürfnissen des Marktes oder der Unternehmer.
\newline

\chapter{Wirtschaft im Wandel}

\begin{enumerate}
\item Entgegenstellung von der Wirtschaftsliberalismus und Marxismus

\begin{tabular}{|c|c|}

\hline Wirtschaftsliberalismus & Marxismus
\\\hline freie Wirtschaft & Wirtschaft vom Staat gesteuert
\\Privateigentum & kein Privateigentum
\\\hline
\end{tabular}

\item Rolle des Staates in der Planwirtschaft und Marktwirtschaft

\end{enumerate}

\chapter{Neoliberalismus}
Das ist die moderne Variante des klassischen Wirtschaftsliberalismus.
\newline
Politische Ziele:

Neutral:

\begin{itemize}
\item Öffnung der Märkte 
\item freier Kapitalverkehr
\item Privatisierung der öffentlichen Güter und Dienstleistungen
\item Deregulierung des Arbeitsmarktes
\item Liberalisierung der Preise
\item Steuerliche Entlastung von Unternehmen und Vermögen
\end{itemize}

Negative:

\begin{itemize}
\item Einschränkung der Sozialleistungen $\rightarrow$ Zerstörung des sozialen Netzes
\item Zurückdrängen der Gewerkschaften
\end{itemize}

Thatcherismus (Groß Großbritannien) | Reaganomics (USA) | ''Heuschrecken Kapitalismus''
\newline
\newline
Privatisierung
Einschränkung der Sozialleistung
Massive Steuersenkung
Zurückdrängen der Gewerkschaften

\section{Wirtschaftstheoretische Erklärungsmehtoden für den Strukturwandel}

Konratierv erkannte das ca. alle 50 Jahre Innovationen einen Wachstumsschub auslösen. Lässt dieser nach, schlägt die Wirtschaft in die Rezession. Die Auf und Abschwünge der Konjunktur nennt man Lange Wellen. Die Basis für diese Langen Wellen ist eine grundlegende technische Innovation. Und diese Innovation bewirkt eine Veränderung der Produktion und bei der Organisation.
\newline
\newline
(Zettel)
\newline
\newline
Basisinovationen:

\begin{itemize}
\item Eisenbahn, Dampfschiffe, Eisen- und Stahlindustrie (a)
\item Bio- und Gentechnik, Mikroelektronik (b)
\item Dampfmaschine, Textil- und Eisenproduktion (c)
\item Elektronik, Automobilindustrie, Petrochemie (d)
\item Elektrizität, Chemieindustrie (e)
\item Informationstechnik (f)
\end{itemize}

Basiseinovationen für mögliche künftige Wellen:

\begin{itemize}
\item Gentechnik
\item Luft- und Raumfahrt
\item (Alternative) Energien
\item Biotechnologie
\item Gesundheit
\end{itemize}

\subsection*{Das magische Vieleck der Volkswirtschaft}

Das Funktionieren einer Volkswirtschaft kann man mit einer Vielzahl von Zahnrädern vergleichen, die ineinander greifen. Bewegt man ein Rad, hat das Auswirkungen auf die anderen z.B. geht es der Wirtschaft schlecht $\rightarrow$ steigt die Arbeitslosenrate, Wirtschaft gut $\rightarrow$ Preise steigen, ...
\newline
\newline
Die Politik greift mehr oder weniger stark in diese Prozesse ein, kann aber nicht den idealen Zustand herstellen. Der Einfluss eines Landes wird immer geringer, da die Wirtschaft zunehmend weltweit organisiert ist.
\newline
\newline
Magisches Vieleck $\rightarrow$

\begin{itemize}
\item Wirtschaftswachstum
\item Geldwertstabilität
\item Hohes Beschäftigungsniveau
\item Außenwirtschaftliches Gleichgewicht
\item Lebensqualität/ Umwelt
\item Gerechtes Einkommen- / Vermögensverteilung
\end{itemize}

Da ein Markt nie perfekt ist, ist es die Aufgabe der Wirtschaftspolitik dort einzugreifen wo der Markt versagt (Deregulierung, Regulierung).

\subsection*{Träger der Wirtschaftspolitik in Österreich}

\begin{itemize}
\item Regierung, Minister, Politische Parteien
\item Gewerkschaft
\item Interessenvertreter der Wirtschaft / Landwirtschaft
\item Lobbys
\end{itemize}

Das jährliche Wirtschaftswachstum misst an der Prozentualen Zunahme des BIP (Bruto Inlands Produkt). Den wellenförmigen Verlauf nennt man Konjunktur. Wachturmrate von unter 2\% gelten als niedrige Wachstum, bereits an der Schwelle zur Stagnation. Ab 5 bis 6\% spricht man von hohen Wirtschaftswachtum.

\subsection*{Hohes Beschäftigungsniveau}
ru
Arbeitslosigkeit ist prinzipiell ein soziales Problem. Sie verursacht durch die Arbeitslosigkeit durch die Arbeitslosenkeitenunterstürzten, ausfallende Steuern und soziale Abgaben. Die Aufgabe der Politik ist der die Arbeitslosigkeit so gering wie möglich zu halten.
Liegt die Arbeitslosigkeit, bei unter zwei Prozent so spricht man von Vollbeschäftigung. In Österreich ist die Arbeitslosenquote niedrieger als in den anderen EU Staaten, allerdings steigt sie auch in Österreich. Verschiedene Gründe geben die Arbeitslosigkeit an.

\begin{itemize}
\item 1.) Konjunkturelle Arbeitslosigkeit: in Phasen der Hochkonjunktur tendiert die Wirtschaft normalerweise zur Vollbeschäftigung
In Zeiten der Rezession tendiert sich zur hohen Arbeitslosigkeit
\item 2.) Strukturelle Arbeitslosigkeit : Tritt besonders in Branchen auf, die schrumpfen (z.B. Textil Industrie) bzw. in denen aufgrund technologischen Erneuerung Arbeitskräfte ersetzt werden (=Technologische Arbeitskraft).
\item 3.) Saisonale Arbeitslosigkeit:  diese tritt besonders in Branchen auf, deren Auftragslage starken saisonalen Schwankungen aussetzt sind (z.B. Bauwirtschaft, Tourismus)
\item 4.) Friktionelle Arbeitskraft: vorübergehende Arbeitslosigkeit zwischen zwei verschiedenen Arbeitsstellen
\end{itemize}
Problem ist dabei ist die Erfassung der Arbeitslosen. Arbeitslose außerhalb der Statistik:

\begin{itemize}
\item vorherige Berufstätig notwendig
\item Schulungen (Umschulungen, Fortbildungen)
\item Vorruhestand 
\item ''Stille Reserve'' (Schüler, Frauen nach der Kindererziehung)
\item ''Wohlstandarbeitslosigkeit'' (Leute, die sich die Arbeitslosigkeit leisten können)
\end{itemize}

Die Messung der Arbeitslosigkeit ist ein statistisches Problem (Probleme bei den Begriffen Erwerbsperson, Erwerbslose, Arbeitslose). Grund sächlich gibt es zwei Methoden, die österreichische und die einheitliche europäische Methode.

\end{document}
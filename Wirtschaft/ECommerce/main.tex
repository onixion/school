\documentclass[a4paper]{article}
\usepackage[left=1.8cm, right=1.8cm, top=2.0cm, bottom=2.0cm]{geometry}
\usepackage[utf8]{inputenc}
\author{Egretzberger Dominik, Porcic Alin}
\title{E-Commerce}

\begin{document} 

\maketitle
\newpage

\tableofcontents
\newpage

\section{Was ist E-Commerce?}

E-Commerce steht für \textit{Electronic Commerce} (also elektronischer Handel) und ist ein Sammelbegriff für das Anbahnen, Abschließen und Abwickeln von Geschäften über das Internet. Für die Anboeter 3und Nachfrager eröffnen sich durch diese Art des digitalen Handels eine Vielzahl unterschiedlicher Möglichkeiten der Information, Kommunikation und Transaktion. E-Commerce als Begriff bezeichnet aber auch allse was geschäftsdienend über das Internet übertragen wird (eine simple E-Mail Anfrage über den Öffnungszeiten zählt genauso zum E-Commerce wie die Bestellung eines Artikels in einem Online-Shop).\\\\

\section{Allgemeines}

Außerhalb des Internet waren die Rechtsbeziehungen von Unternehmen oder auch einzelner Personen vor allem dadurch geprägt, dass sie regelmäßig aus einem direkten Kontakt mit anderen entstanden. Dies führte in der Regel zu einem eingeschränkten Aktionsradius und damit auch meist zu einer eindeutigen Anwendbarkeit des Rechts des Staates, in dem man sich gerade befand. Nur selten kam es zu Verflechtungen mit anderen Rechtssystemen.\\
Mit dem Siegeszug des E-Commerce hat sich dieses Bild grundlegend geändert. Das Internet als weltweites Kommunikationsmedium lässt Entfernungen bedeutungslos werden. Dem Anbieter von Waren- oder Dienstleistungen im Internet stehen nun Nachfragende auf der ganzen Welt gegenüber. Es kommt so zu einer Vielzahl von internationalen Rechtskontakten, ohne dass dies den Beteiligten immer bewusst ist. Hinzu kommen neue Arten der Willenserklärung, wie E-Mail oder Mausklick. Solange derartige Rechtsbeziehungen zur Zufriedenheit aller abgewickelt werden, ist das eine tolle Sache. Kommt es aber zu Leistungsstörungen, zeigt sich, dass das gewohnte Instrumentarium versagt. Das beginnt schon mit der Eruierung des Vertragspartners. Wer steckt eigentlich hinter der Website xy und wo ist sein Wohnsitz oder Unternehmenssitz? Wo kann man klagen? Nach welchem Recht richten sich die geltendzumachenden Ansprüche? Kein Wunder, dass man bald erkannt hat, dass man hier ohne überregionale Regelungen nicht weiterkommt.

\section{Arten von E-Commerce}

\begin{itemize}
\item Business to Business-Commerce (B2B): z.B. Online-Rohstoffeinkauf und Bestellung
\item Business to Consumer-Commerce (B2C): z.B. Online-Buchbestellung, digitale Broschüren
\item Business to Administration-Commerce (B2A): z.B. Online-Steuerabwicklung
\item Consumer to Consumer-Commerce (C2C): z.B. Kleinanzeige im Internet, Onlineauktion
\item Administration to Consumer-Commerce (A2C): z.B. Online-Steuererklärung
\item Administration to Administration-Commerce (A2A): z.B. Online-Personalabrechnungen, digitale Transaktionen
\item Business to Employee-Commerce (B2E): z.B. Reisekosten- oder Diätenabrechnungen

\end{itemize}

\section{Vor- und Nachteile von E-Commerce}

\subsection{Vorteile für den Anbieter}

\begin{itemize}
\item 24/7 Verfügbarkeit
\item Weltweite Erreichbarkeit
\item Schnelle Aktualisierung
\item Erstellung von Kundenprofilen
\item Chancengleichheit auch für kleinere Unternehmen
\item Zeit- und Kosteneinsparungen möglich
\item Wegfall des Standortnachteils
\end{itemize}

\subsection{Nachteile für den Anbieter}
\begin{itemize}
\item Hohe Vorlaufkosten
\item Marktanteile können leichter verloren gehen
\item Anonyme, unbekannte Kunden
\end{itemize}


\subsection{Vorteile für den Nachfragenden}
\begin{itemize}
\item Zeit- und Kostenersparnis
\item Stressfreies bzw. gemütliches Einkaufen
\item Unabhängig von den normalen Öffnungszeiten
\item Direkte Preisvergleiche möglich
\item Internationales Einkaufen möglich
\item Anonymes Schuppern und Vergleichen möglich
\end{itemize}

\subsection{Nachteile für den Nachfragenden}
\begin{itemize}
\item Verzicht auf soziale Einkaufserlebnisse
\item Mögliche Datenschutzmissachtung des Betreibers
\item Artikel können nicht angefasst werden, angesehen bzw. ausprobiert werden
\item Lieferzeiten - eingeschränkte Verfügbarkeit
\end{itemize}

\newpage
\section{Rechtlicher Aspekte}

\subsection{AGB}

\subsubsection{Was ist die AGB?}

AGBs sind Vertragsklauseln, die zur Verwendung in einer Vielzahl von Verträgen vorformuliert wurden und von einer Vertragspartei (Verwender) der anderen Vertragspartei vorgelegt werden. Klauseln, die zu den Allgemeinen Geschäftsbedingungen zählen, finden sich in nahezu allen Verträgen, seien es Arbeits-, Kauf-, Werk-, Miet-, Telefondienstleistungsverträge und in vielen anderen mehr. Die Verwendung von Allgemeinen Geschäftsbedingungen (AGB) ist nur dann sinnvoll, wenn diese auch zum Vertragsinhalt werden. Fehlt ein deutlicher Hinweis auf AGB oder kann der Vertragspartner nicht auf zumutbare Weise (z.B. weil die AGB unleserlich klein gedruckt wurden) Einsicht nehmen, werden AGB von vornherein nicht Vertragsinhalt.\\\\
Die AGBs haben zumeist das Ziel, die gesetzlichen Vorgaben zugunsten des Verwenders zu modifizieren, den Vertragsschluss und die -abwicklung zu erleichtern und zu standardisieren. Insbesondere wenn vielfach gleichartige Verträge geschlossen werden, ist die Verwendung von AGBs eine wesentliche Erleichterung für die Parteien, da dann nicht mit jedem Kunden auch die Rahmenbedingungen erneut ausgehandelt werden müssen. Ohne solche Klauseln wäre ein Massengeschäft (z. B.: Mobilfunkverträge, Versicherung, Elektronikhandel) nicht zu handhaben. Weiterhin dienen AGBs auch der Vermeidung von Streitigkeiten, da sie viele der ansonsten unausgesprochenen Fragen schriftlich regeln.\\\\
Es ist ratsam, dem Partner die Geltung der AGB zweifelsfrei bekanntzugeben, ihm diese nach Möglichkeit vor seiner Unterschrift lesen und sich dieses Lesen auch bestätigen zu lassen. Zumindest sollte vor der Unterschrift ein deutlicher Hinweis auf die Geltung von AGB vorhanden sein. Die Übermittlung von AGB auf Rechnungen, Lieferscheinen oder dergleichen ist grundsätzlich wirkungs- und damit sinnlos.

\subsubsection{Häufigsten AGB-Fallen}

\begin{itemize}
\item \textbf{Verfallsdatum}: Ob eine Schokoladen-Massage für die Liebste oder ein Golf-Schnupperkurs für den Vater: Gerade bei Präsenten mit Eventcharakter greifen Kunden immer häufiger auf Geschenkgutscheine zurück. Dabei wird übersehen, dass Papier in diesem Fall keinesfalls geduldig ist: Die meisten Geschenkgutscheine sind zeitlich begrenzt. Wird die Einlösefrist versäumt, ist das Geld futsch. Wer dagegen klagt, hat zwar gute Chancen auf Aufhebung der Frist, aber wer zieht schon wegen 50 oder 100 Euro vor Gericht? So können Sie sich schützen: Achten Sie darauf, dass der Gutschein mindestens sechs, besser noch zwölf Monate gültig ist. Kürzere Fristen entsprechen nicht dem üblichen Standard - außer, es handelt sich um ein zeitlich begrenztes Event. Genau hinschauen sollten Sie auch beim Thema Fristverlängerung, Übertragung des Gutscheins an Dritte und Restgeldauszahlung. Sind die Punkte in den AGB nicht vorhanden oder nur schwammig formuliert, nachfragen und entsprechende Zusagen schriftlich geben lassen.

\item \textbf{Versandkosten}: Auf die Versandkosten bei der Bestellung achten die meisten von uns, beim Thema Rückgabe haben sie allerdings die wenigsten auf dem Radar. Die entsprechenden Informationen findet man in den AGB unter 'Widerrufs- oder Rückgaberecht'. Wird ein "Widerrufsrecht" eingeräumt, muss der Händler die Kosten für die Rücksendung von Waren im Wert von mehr als 40 Euro nur dann übernehmen, wenn der Kunde den Kaufpreis bereits gezahlt oder zumindest angezahlt hat. Ist beides nicht der Fall oder liegt der Warenwert unter 40 Euro, bleibt der Kunde meist auf den Kosten sitzen. (Ausnahmen gibt es nur bei falsch gelieferter oder kaputter Ware.) Nur wenn der Händler dem Kunden in den AGB ausdrücklich ein 'Rückgaberecht' einräumt, muss er auch für die Rücksendekosten aufkommen. So können Sie sich schützen: Da hilft nur, schon vor dem Kauf genau hinzuschauen und im Zweifel lieber bei einem Anbieter zu bestellen, der die Versandkosten auch bei der Produktrückgabe übernimmt.

\item \textbf{Änderung von AGB}: Insbesondere wenn es um langfristige Dienstleistungsverträge geht, muss man als Kunde vor nachteiligen Änderungen der AGB auf der Hut sein. So kann der Dienstleister seine Preise auch nach Vertragsabschluss noch ändern - er muss den Kunden lediglich darüber informieren. Bekommt der Kunde die neuen AGB zugesandt, ist die Informationspflicht erfüllt. Wer den neuen Bedingungen innerhalb einer bestimmten Frist nicht widerspricht, gibt sein Einverständnis dazu. Unerfahrene Kunden nehmen die neuen AGB zwar zur Kenntnis, gehen aber häufig davon aus, dass sie von ihnen gar nicht betroffen sind - ihr Vertrag basiert ja schließlich auf den alten. Ein Irrtum, der teuer werden kann. So können Sie sich schützen: Schauen Sie sich die neuen AGB genau an und vergleichen Sie diese mit den alten. Sollten Sie Nachteile erkennen, dann widersprechen Sie schriftlich, nur so bleiben die alten AGB für Sie gültig.

\item \textbf{Ausschluss von Widerrufs- oder Rückgaberecht}: Die meisten von uns haben zwar schon mal gehört, dass man Verträge innerhalb von zwei Wochen widerrufen oder die Ware zurückgeben kann. Viele Verbraucher wissen aber nicht, dass der Gesetzgeber auch einen Ausschluss des Widerrufs- oder Rückgaberechts in den allgemeinen Geschäftsbedingungen (AGB) zulässt. So beispielsweise bei individuell angefertigten oder verderblichen Waren und bei Produkten, die nicht zurückgesandt werden können (z. B. Heizöl oder Benzin). Warum die Ausnahme auch für Pauschalreisen, Tickets und Hotelbuchungen gilt, kann der Kunde zwar nicht nachvollziehen - hinnehmen muss er es trotzdem. Auch wer Software, CDs und DVDs "entsiegelt", verwirkt laut AGB das Widerrufs- und Rückgaberecht. So können Sie sich schützen: Die AGB genau durchlesen, nachfragen - und notfalls lieber woanders buchen. Bei Software, CDs und DVDs: Vom Händler schriftlich erklären lassen, wie man die Ware prüfen kann, ohne die Originalverpackung zu beschädigen.

\item \textbf{Zeitschriften-Abo}: Auch für Zeitungs- oder Zeitschriftenabonnements gelten spezielle Regeln, die kaum ein Verbraucher kennt. So können Kunden zwar das Zeitschriften-Abo, das ihnen telefonisch oder an der Haustür verkauft wurde, mit Verweis auf die Vorschriften zu Fernabsatzverträgen und Haustürgeschäften widerrufen. Wer aber via Internet, Mailing oder mithilfe eines Aboformulars schriftlich bestellt, kann den Vertrag nicht mehr widerrufen. Ausnahme: Wenn das Abo zum ersten möglichen Kündigungstermin mehr als 200 Euro kostet - die meisten Gazetten liegen deutlich darunter. Ein Widerruf wird außerdem auch bei Probeabonnements und im Voraus gezahlten Jahresabos ausgeschlossen. So können Sie sich schützen: Schutz bietet hier nur der Informationsvorsprung - das genaue Durchlesen der AGB vor Vertragsabschluss.

\item \textbf{Frühbucherrabatt}: Die Freude über den saftigen Frühbucher-Rabatt kann ebenfalls in teueren Ärger umschlagen. Denn Veranstalter und Airlines dürfen den Preis für eine Pauschalreise oder einen Flug nachträglich erhöhen, wenn sie sich dieses Recht vertraglich mit einer Preisänderungsklausel vorbehalten haben. So dürfen höhere Preise für Kerosin und Sprit, gestiegene Hafen- und Flughafengebühren und Wechselkursschwankungen an den Kunden weitergegeben werden. So können Sie sich schützen: Wer kurzfristiger bucht, ist auf der sicheren Seite. Denn der Kunde muss spätestens drei Wochen vor Abreise über die neuen Preise informiert werden. Wird die Preiserhöhung später bekannt gegeben, muss der Kunde nicht zahlen.

\item \textbf{Kredit-Zusatzkosten}: Den neuen Plasma-Fernseher sofort mitnehmen und in kleinen Raten über einen Zeitraum von bis zu drei Jahren abzahlen - bei null Prozent Zinsen! Ein schönes Angebot, wenn da nicht das Kleingedruckte wäre. Denn die Werbung verrät dem Kunden natürlich nicht, dass er sich erst einer Bonitätsprüfung unterziehen muss, bevor er das Angebot tatsächlich wahrnehmen kann. Und wenn zu '0 \% Zinsen' noch die 'einmalige Kontoführungsgebühr' und die Kosten für die bei manchen Anbietern angeblich vorgeschriebene Restschuld-Versicherung dazukommen, kann das begehrte Produkt ganz schön teuer werden. Natürlich arbeiten nicht alle Anbieter mit diesen Tricks - aber doch so viele, dass sich das genaue Durchlesen der allgemeinen Geschäftsbedingungen sicher bezahlt macht. So können Sie sich schützen: Fragen Sie genau nach, ob und welche zusätzlichen Gebühren bei einem Ratenkauf anfallen. Interessant ist für Sie nicht, wie der Händler die Zusatzkosten nennt, sondern ob es welche gibt und wie hoch der Unterschied zum Sofortkauf ist.

\item \textbf{Günstige Kredite}: Wer sich sein Plasma-TV nun lieber nicht vom Händler, sondern vom Kredithai finanzieren lassen will, ist nicht gut beraten. Vor allem von 'Instituten', die dafür werben, Kredite ohne Schufa-Prüfung zu vergeben, sollte man lieber die Finger lassen. Diese Darlehen werden extrem hoch verzinst - das ist den meisten Verbrauchern inzwischen bekannt. Was Kunden aber häufig übersehen: Hinter diesen Angeboten verstecken sich meist selbstständige Kreditvermittler, die für ihre Arbeit hohe Gebühren verlangen, sodass nach Abzug von 'Bearbeitungsgebühren' und Provisionen nur noch ein Bruchteil des Kredites tatsächlich ausgezahlt wird. Über diese 'versteckten' Gebühren wird der Kunde informiert, sie stehen in den AGB. So können Sie sich schützen: Vor Vertragsunterzeichnung um eine schriftliche Auflistung aller anfallenden Kosten und Gebühren bitten. Noch besser: Kredite lieber nur bei einem Kundenberater abschließen, mit dem man bereits gute Erfahrungen gemacht hat, etwa bei der Hausbank.

\item \textbf{Autokauf}: Privatpersonen können mit einem Satz im Vertrag das Thema "Garantie und Gewährleistung" komplett ausschließen - ein Händler kann und darf das hingegen nicht. Wer also beim Autokauf von privat den hier inzwischen gängigen Ausschluss der Gewährleistung akzeptiert, muss davon ausgehen, dass der Verkäufer für etwaige Mängel nicht mehr haften muss. Das gilt allerdings nicht, wenn die Mängel "arglistig" verschwiegen wurden, dem Anbieter also vor Vertragsabschluss bekannt waren. Interessenten fühlen sich damit abgesichert - und übersehen, dass die Beweislast bei ihnen liegt und sie dem Verkäufer erstmal nachweisen müssen, dass ihm der Mangel bekannt war. So können Sie sich schützen: Da der Ausschluss von Garantie und Gewährleistung bei privaten Verträgen quasi schon Standard ist, wird der Verkäufer freiwillig sicher nicht darauf verzichten. Wer sich vor Überraschungen schützen will, sollte das Auto deshalb vor dem Kauf von einer Fachwerkstatt durchchecken lassen.

\item \textbf{Bestellungen im Ausland}: Wer die neueste Technik im Ausland bestellt, kann durchaus ein gutes Geschäft machen: Außerhalb Deutschlands werden viele Produkte nicht nur früher, sondern häufig auch billiger angeboten. Doch die Überraschungen lauern im Kleingedruckten: So rechnen die wenigsten Kunden damit, dass sie nicht nur das begehrte Produkt, sondern zusätzlich noch eine Einfuhrumsatzsteuer oder Zollsätze bezahlen müssen. Versand- und Lieferkosten werden ebenfalls oft unterschätzt: Wer nicht aufpasst, muss sein Schnäppchen am Ende doppelt teuer bezahlen. So können Sie sich schützen: Achten Sie darauf, ob der Händler die Zusatzkosten angibt. Wer seine Ware für den Versand ins Ausland anbietet, sollte die dazugehörigen Versand- und Lieferkosten benennen können. Anders sieht es beim Thema Zoll aus, hier müssen Sie sich schon selbst erkundigen. Seriöse Händler werden Sie aber darauf hinweisen, dass mit solchen Gebühren zu rechnen ist.

\end{itemize}

\newpage
\section{Wichtige rechtlichen Anforderungen für einen eigenen Online-Shop}

Wer einen Online Shop betreibt, muss sich über viele Dinge Gedanken machen. Manche bringen Geld, manche können Geld kosten. Das Risiko einen Shop zu betreiben, der rechtlich nicht sauber aufgesetzt ist, kann im schlimmsten Falle zu einer Abmahnung führen. Diese kostet in der Regel eine drei-vierstellige Summe. Vorsicht also! Es gilt natürlich wie so oft: wo kein Kläger, da kein Richter. Bevor man abgemahnt werden kann, muss sich erst ein Wettbewerber oder Verbraucher an den inkorrekt umgesetzten Vorgaben des Online Shops stören. Deutlich ruhiger schläft man aber, wenn man beachtet, was der Gesetzgeber vorgibt.\\\\
Dass zur Grundausstattung einer gewerblichen Website in Deutschland ein Impressum gehört, dürfte den meisten Website-Betreibern klar sein. Dieses muss leicht erkennbar, unmittelbar erreichbar und ständig verfügbar sein. Auch die Angabe einer Telefonnummer im Impressum wird ab 13. Juni 2014 zur Pflicht für Online-Händler. Außerdem darf auch die Datenschutzbelehrung nicht fehlen, falls Sie Nutzerdaten tracken wie z.B. mit Google Analytics.\\\\
Was darf nicht fehlen:

\begin{itemize}
\item \textbf{Preisangaben}: Wer seine Geschäfte mit Verbrauchern macht, muss zwingend den Bruttopreis der Ware angeben. Und es ist des Weiteren darauf hinzuweisen, dass in dem Preis Umsatzsteuer enthalten ist. Wenn Versandkosten anfallen, müssen diese ebenfalls angegeben werden. In Deutschland hat sich in Online Shops die Darstellung 'inkl. MwSt. zzgl. Versandkosten' bewährt. Diese Informationen müssen direkt beim Produkt stehen.
\item \textbf{Hinweise zu Lieferzeiten}: Der Verkäufer muss eine Angabe treffen, wie lange ab der Bestellung dauern wird, bis der Artikel geliefert wird. Dabei muss nicht der exakte Tag der Lieferung angegeben werden (das ist i.d.R. auch schwer zu sagen), sondern ein Zeitraum. 'Lieferzeit ca. 3 bis 5 Tage' wäre somit eine gültige Bezeichnung laut Rechtsanwalt Martin Rätze. Zu finden muss diese Information auf der Produktseite sein. Falls für unterschiedliche Länder unterschiedliche Lieferzeiten gelten, dann sollte auf eine Übersichtsseite verlinkt werden, wo die entsprechenden Informationen zu finden sind. Lediglich eine Angabe zur 'Versandbereitschaft' (z.B. 'sofort lieferfertig') zu treffen reicht im übrigen nicht aus.
\item \textbf{Darstellung des Warenkorbs}: Auf der Seite des Warenkorbs muss sowohl der Preis des Produktes zu sehen sein, wie auch die Versandkosten. Die einzelnen Schritte bis zum Abschluss der Bestellung müssen ersichtlich sein, z.B. in Form einer Timeline. Außerdem muss die Aufschrift des Bestell-Buttons deutlich zeigen, dass man die Bestellung abschließt. Es ist nicht in Ordnung diesen z.B. nur mit 'Weiter' zu beschriften.
\item \textbf{Datenabfrage bei Adresseingabe}: Bei den Daten, die man jetzt abfragt, darf gemäß dem Datenvermeidungsprinzip die Telefonnummer kein Pflichtfeld sein. Es sei denn, diese ist zur ordnungsgemäßen Bearbeitung der Bestellung unbedingt nötig. Bei der Länderauswahl dürfen nur die Länder auswählbar sein, für welche auch Versandkosten angeben wurden.
\item \textbf{Zahlungsarten}: Prinzipiell muss ein gängiges Zahlungsmittel ohne zusätzliche Gebühren angeboten werden. Fallen für bestimmte Zahlungsarten zusätzliche Gebühren an (z.B. PayPal), so muss darauf hingewiesen werden. Zum einen auf einer allgemeinen Informationsseite, zum anderen auch im Bestellprozess. Es ist nicht erlaubt, zusätzliche Aufschläge auf bestimmte Zahlungsmittel zu erheben.
\item \textbf{Pflichtangaben Bestellseite}: Hier muss man dem Verbraucher einen kompletten Überblick über die bestellten Produkte, Versandkosten und eventuelle Zusatzkosten geben. Ebenfalls muss der Gesamtpreis, die Zahlungsart und die Liefer- bzw. Rechnungsanschrift enthalten sein. Laut Martin Rätze muss der Kunde auch darauf hingewiesen werden, wie man seine persönlichen Daten nochmal verändern kann (z.B. wegen eines Eingabefehlers). Falls diese Information fehlt, verlängert sich die Widerrufsfrist und es kann ein Grund für eine Abmahnung gegeben sein.
\item \textbf{Hinweis auf das Widerrufsrecht und AGB}: Vor Abschluss der Bestellung muss der Verbraucher deutlich auf das Widerrufsrecht und die AGB hingewiesen werden. Der Händler ist verpflichtet ein Widerrufsformular zur Verfügung zu stellen (online oder offline). Wenn die beiden Texte nicht vollständig auf der Seite dargestellt werden können (was meistens der Fall ist), müssen die beiden Worte auf die ausführlichen Erklärungen verlinkt werden. Bei der Widerrufsbelehrung ist aber entscheidend, dass sie dem Kunden schriftlich zugesendet wird (z.B. in der Bestellbestätigung per E-Mail). Es ist nicht ausreichend den Text lediglich auf der Website zu haben.

\item \textbf{Bestellbestätigung}: Eine erfolgte Bestellung muss unverzüglich auf elektronischem Wege bestätigt werden, was in der Regel per E-Mail passiert.

\item \textbf{Die 'Button-Lösung'}: Da Verbraucher immer häufiger Probleme durch Abzock-Firmen und deren Abo-Fallen bekamen, wurde in Deutschland die sogenannte Button-Lösung eingeführt. Wer diesen Anforderungen nicht nachkommt, hat das Problem, dass kein gültiger Kaufvertrag zustande kommt! Sehen wir uns die Anforderungen im Einzelnen an: Der Verbraucher muss eindeutig erkennen können, dass die Bestellung eine Zahlungsverpflichtung nach sich zieht. In Ordnung ist zum Beispiel 'zahlungspflichtig bestellen' oder auch 'kostenpflichtig bestellen'. Nicht ok ist beispielsweise: 'bestellen', 'weiter' oder 'Bestellung abschließen'.


\end{itemize}

\end{document}

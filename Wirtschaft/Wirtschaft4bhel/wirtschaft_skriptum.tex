\documentclass[a4paper]{report}
\usepackage[utf8]{inputenc}
\usepackage[german]{babel}
\usepackage{graphicx}
\usepackage[font]{}

\begin{document}

\chapter{Einführung}

Die  vier Wirtschaft Subjekten:
\newline
\newline
\newline
\newline
\includegraphics[scale=0.8]{image/image1.png}
\newline
\newline
\newline
\newline
Die Wirtschaftstätigkeit eines Landes wird zahlenmäßig erfasst, wobei verschiedene Ströme festgestellt werden. Zwischen den vier Wirtschaft Subjekten kommt es zu Transaktionen, Austausch von Gütern. Es gibt zwei gegenläufige Ströme:

\begin{itemize}
\item Realer Strom: Güter (Waren) und Dienstleistung
\item Monetärer Strom: Geld (Geldstrom)
\end{itemize}

Am Geldstrom wird das Volkseinkommen gemessen, am Güterstrom das Sozialprodukt. Das Sozialprodukt ist ein genereller Maßstab für die Wirtschaftskraft eines Landes. Je größer es ist, desto mehr kann im Allgemeinem verbraucht werden und desto größer ist der rechnerischer Wohlstand der Bevölkerung so fern dieser einigermaßen gleich verteilt ist.

\chapter{Güter}

Einteilung der Güter nach der Verfügbarkeit:

\begin{itemize}
\item öffentliche Güter (unbegrenzt vorhanden)
\item knappe Güter (Sachgüter, Dienstleistung, Rechte, Eigentumsrecht, ...)
\end{itemize}

Einteilung der Güter nach Verwendung:

\begin{itemize}
\item Konsumgüter
	\begin{itemize}
	\item Verbrauchsgüter (für einmaligen Gebrauch z.B. Nahrung, ...)
	\item Gebrauchsgüter (für mehrmaligen Gebrauch z.B. Auto, ...)
	\end{itemize}
\item Produktionsgüter (Güter mit den sich andere Güter herstellen lassen können)
\end{itemize}

\chapter{Produktionsfaktoren}

\section{Produktionsfaktoren}

\begin{itemize}
\item Boden
\item Arbeit
\item Wissen (Know-How)
\item Boden
\end{itemize}




\chapter{Taylorismus}

Wenn man einen komplexen Arbeitsprozess in möglichst viele kleinere Prozesse zerteilt, spricht man von Taylorismus. Zudem trennt man räumlich und personell die ausführende Arbeit mit der dispositiven Arbeit (Weisungsbefugnis).

\section{Vor- und Nachteile:}

\subsection{Vorteile:}

\begin{itemize}
\item Arbeiter benötigen nicht spezielles Wissen (billige Arbeitskräfte) oder lange Einarbeitungsphase
\item Arbeiter können leicht ersetzen werden
\item Transparenz in der Produktion und auch leichte Fehlersuche im Arbeitsprozess
\end{itemize}

\subsection{Nachteile:}

\begin{itemize}
\item Arbeiter langweilen sich, Monotonie $\rightarrow$ keine Motivation (kann zu Streiks führen)
\item körperliche Schäden $\rightarrow$ einseitige Belastung (z.B. Räder stemmen)
\item schlechtes Arbeitsklima, da es zu keiner Kommunikation zwischen den Arbeitern möglich ist
\item sinkende Lern- und Anpassungsmöglichkeiten an neue Aufgaben
\end{itemize}

\section{Andere Arbeitsformen}

\begin{itemize}
\item job rotation: Nach einem festen System werden regelmäßig die Arbeitsplätze getauscht. Die Struktur der Aufgaben wird nicht angerührt.
\item job enlargement: Zusätzliche Aufgaben werden zusammengeführt.
\item job enrichment: qualitative Ausweitung der Aufgaben, Eigenverantwortung
\item Team Arbeit (Projekt): große Motivation
\end{itemize}

\chapter{Wirtschaftssektoren}

\begin{itemize}
\item primärer Wirtschaftssektor: Urgewinnung
\item sekundärer Wirtschaftssektor: Produktion
\item tertiärer Wirtschaftssektor: Dienstleistung
\item quartärer Wirtschaftssektor: IT
\end{itemize}

\chapter{Arbeitslosigkeit}

\section{Definition}

Eine Person ist dann Arbeitslos, wenn die Person arbeitsfähig und arbeitswillig ist, sie schon einmal gearbeitet hat und nach Arbeit sucht.

\section{Arten der Arbeitslosigkeit}

\begin{itemize}
\item konjunkturelle Arbeitslosigkeit: die allgmeine Nachfrage nach Gütern und Dienstleistungen geht zurück $\rightarrow$ Arbeitskräfte werden entlassen $\rightarrow$ weitere Kaufkraft geht verloren.
\item strukturelle Arbeitslosigkeit: Verschiebung der Wirtschaftssektoren
\item friktionelle Arbeitlosigkeit: der Zeitraum ohne Arbeit den Arbeitsplätzen
\item saisoneale Arbeitslosigkeit: Seasonarbeit (z.B. Skifahren)
\item verdeckte Arbeitslosigkeit: betrifft Personen, die den Neueinstieg oder den Wiedereinstieg planen (z.B. Schüler, Frau nach Geburt)
\end{itemize}

\chapter{Konjunktur Theorie}

\begin{itemize}
\item John Majuard Kaynes (1883-1946)

In den 30er Jahren kam es Aufgrund der großen Weltwirtschaftskrise zu Massenarbeitslosigkeit. Kaynes empfahl der britischen Regierung, sich bei den Banken Geld zu leihen und damit Aufträge an die Industrie zu finanzieren. Die aufgenommenen Kredite könne man dann in der folgenden Boomphase (hohe Beschäftigung $\rightarrow$ reichliche Steuereinnahmen) wieder zurückzahlen.

\item Milton Friedman (1912-2006)

In den 60er Jahren feierte der Fiskalismus glanzvolle Erfolge. Viele glaubten, man könne die Wirtschaft nach belieben ''ankurbeln'' oder ''bremsen''. In den 70er kamen zweifel auf $\rightarrow$ wirtschaftliche Stagnation, hohe Arbeitslosigkeit bzw. Inflation. Friedman war der schärfste Kritiker des Kaynsianismus. Seine Meinung nach gehört der ganze ''Sozialklingbling'' (Kinder- oder Wohngeld) abgeschafft. Er leugnet zwar nicht die Möglichkeit von Arbeitslosigkeit, weil sich nicht alle Arbeitnehmer an veränderte Strukturen anpassen können oder wollen. Außerdem muss der Staat sich das zur Ausgaben finanzierende benötigte Geld auf dem Kapitalmarkt leihen $\rightarrow$ Zinsen steigen und private Investoren werden zurückgedrängt.
\end{itemize}

\chapter{Angebot und Nachfrage}

Nachfrage:
\newline
\newline

BILD

Die Nachfrage hängt ab von:

\begin{itemize}
\item Nutzen des Gutes
\item Einkommen $\rightarrow$ Kaufkraft
\item Qualität
\item Verfügbarkeit
\item Wertschätzung
\item Trend
\item Preis von Substitutionsgüter
\end{itemize}

Angebot:

BILD

Das Angebot hängt ab von:

\begin{itemize}
\item Produktionsbedienungen
\item Menge, die angeboten werden sollen
\item Kosten
\item Technologie
\end{itemize}

Beide Diagramme übereinander legen:
\newline
\newline

Markt Gleichgewicht.

\newpage

\section{Das Recht}

Es gibt Gesetze, die man vereinbart hat, um für Ordnung zu sorgen.

\begin{itemize}
\item \textbf{objektives Recht}: ist für die Gemeinschaft verbindliche Ordnung $\rightarrow$ Zusammenleben der Menschen $\rightarrow$ Durchsetzung durch Zwang; ist bei allen Menschen gleich
\item \textbf{subjektives Recht}: ist das Recht das jedem Einzelnen von uns zusteht z.B. Eigentumsrecht, Erbrecht, usw.; Eigentumsrecht und Erbrecht ist individuell
\end{itemize}

Öffentliches Recht  $\rightarrow$ Beziehung zwischen Einzelperson und Staat
\newline
Privates Recht $\rightarrow$ Beziehung zwischen Privatpersonen

\subsection{Gruppenarbeit}

\subsubsection{Stufenbau der Rechtsordnung}

\begin{itemize}
\item EU
\item Verfassung
\item Gesetze z.B. Schulpflicht (Leistungsbeurteilung, Recht auf Benotung, Pflicht der Lehrer)

	\begin{itemize}
	\item Bundesgesetz z.B. Höchstgeschwindigkeit (ABGB; Allgemeine Bürgerliches Gesetz Buch; Grundlagen aller Gesetzte)
	\item Landesgesetz z.B. Jugendschutzgesetz
	\end{itemize}

\item Verordnungen: sind dazu da die Gesetze in die Praxis umzusetzen z.B. wie der Lehrer unterrichten muss, ob er Schularbeiten machen darf oder nicht
\item Bescheid: z.B. positives Bescheid (Baubescheid, ...); negative Bescheid (Führerschein Entzug, ...)
\item Urteile
\item Strafen
\end{itemize}

\subsubsection{Grundprinzipien der Verfassung}

\begin{itemize}
\item Grundprinzipien:

	\begin{itemize}
	\item \textbf{demokratische Prinzip}
	
			\begin{itemize}
			\item indirekte
			\item direkte
			\item Volksbefragung (Ergebnis hat keine Auswirkung)
			\item Volksbegehren (braucht es mindesten 100.000 Stimmen in Form von Unterschriften, dann geht es in den National Rat)
			\item Volksabstimmung (Änderung der Verfassung) z.B. Österreich beitritt zur EU
			
			\item Wahlrecht
				\begin{itemize}
				\item aktives Wahlrecht (16 Jahren)
				\item passives Wahlrecht (19 Jahren)
				\end{itemize}		
			\end{itemize}				
	
	\item \textbf{republikanisches Prinzip}
	
		\begin{itemize}
		\item das Staatsoberhaupt ist der Bundespräsidenten (kann einmal wiedergewählt werden; maximal 12 Jahre)
		\end{itemize}			
	\item \textbf{bundesstaatliche Prinzip}: Gesetzgebung und vollziehen sind zwischen Bund und Land aufgeteilt $\rightarrow$ Kompetenzverteilung, wobei die wichtigsten Staatsaufgaben den Bund zugewiesen werden z.B. die Finanzen.
	\item \textbf{liberales Prinzip}:die staatliche Verwaltung darf nur auf Grundlagen der Gesetze ausgeübt werden
	\item \textbf{reststaatliches Prinzip}: welchen Aufbau haben wir in Österreich? Legislative, Exekutive, Judikative. Gewaltentrennung!
	\end{itemize}
\end{itemize}

\begin{itemize}
\item Zwingendes Recht $\rightarrow$ Recht/ Pflicht das jeder machen muss z.B. Schulpflicht

\item Nachgiebiges Recht $\rightarrow$ Gesetzliche Vorschrift, die durch Parteienänderung abgeändert werden.

\end{itemize}

Personen:

\begin{itemize}
\item natürlichen Personen
\item juristischen Personen $\rightarrow$ Zusammenschluss von mehreren Personen, die nach außen hin eine Einheit bilden und ein Ziel verfolgen z.B. Kapitalgesellschaften (AG und Gesmbh), Vereine, Gebietskörperschaften (Bund, Land, Gemeinde). 
\end{itemize}

Über welche Fähigkeiten verfügen diese Personen? | Rechtsfähigkeit = Trägen von Rechte und Pflichten zu sein. Die sogenannte Handlungsfähigkeit ist die Fähigkeit der Handlungen $\rightarrow$ das man für sich selber Verantwortlich ist (hängt ab vom geistlichen Zustand ab).
\newline
\newline
Unterteilung der Handlungsfähigkeit

\begin{itemize}
\item Geschäftsfähigkeit | Abhängig vom Alter und der geistlichen Reife
\item Deliktsfähigkeit  | Abhängig vom Alter und der geistlichen Reife (ist man am 14ten Geburtstag); man wir für seine Taten zur Verwantwortung gezogen.
\end{itemize}

Altersstufen:

\begin{enumerate}
\item 0 bis 7 ... Kinder (bis zum Vollenden siebten Lebensjahr) (beschränkt Geschäftsfähig)
\item 7 bis 14 ... unmündige Minderjährigen (beschränkt Geschäftsfähig; sie können sehr Wohl Rechte erwerben, ein zu ihrem Vorteil gemachtes Versprechen)
\item 14 bis 18 ... mündige Minderjährigen (ihnen räumt das Gesetzt eine weitergehende Geschäftsfähigkeit ein, d.h. sie können sich selbständig Vertraglich zu Dienstleistung verpflichten, Ausnahme: Lehrverträge; der Mündige dürfen über ihr Einkommen frei verfügen, sofern die Befriedigung der Lebensbedürfnisse nicht gefährdet wird)
\item größer 18 ... volljährig
\end{enumerate}

1 bis 2 sind nur teilweise Geschäftsfähig.

\section{Gesetzliche Vertretung}

Personen, die nicht handlungsfähig sind, bedürfen einen gesetzlichen Vertreter. Falls die Eltern nicht vorhanden sind, ist der gesetzlich Vertreter der Vormund (=Opa,Oma bzw. Personen zu denen ein nahes Verhältnis besteht). Bei mangelten Geisteskraft $\rightarrow$ Sachwalter; vertritt Personen die ihre Angelegenheiten nicht ohne Gefahr eines Nachteils für sich selbst besorgen können. 
\begin{itemize}
\item Bei einzelnen Angelegenheiten (z.B. geistige Behinderung und Testament)
\item bestimmter Bereich (finanzieller Bereich) (z.B. alle Angelegenheiten einer geistlich behinderten Personen)
\end{itemize}

Sachwalter:

\begin{itemize}
\item Familienangehörige
\item bei rechtlichen Angelegenheiten (Interessenkonflikten = dann übernimmt ein Rechtsanwalt)
\end{itemize}

\section{Sachrecht}


\textbf{Ergänzung}: Innehabung: Man hat keinen Besitzwillen es geht ausschließlich darum die Sache zu Verwahren bzw. zu transportieren (Garderobenfrau) bzw. Paketdienst, Postbote, ... Die Sachen dürfen nicht benutzt werden.

Besitzer: ist derjenige, der die Sache willentlich besitzt und nicht bereit ist sie wieder herzugeben.

Arten von Besitz:

\begin{itemize}
\item rechtmäßiger Besitz: gültiger Rechtstitel (Vertrag; mündlicher Vertrag; ich leihe dir mein Handy)
\item unrechtmäßiger Besitz: kein gültiger Rechtstitel (Dieb)
\item redlichen Besitzer: wer glaubt, einen gültigen Rechtstitel zu besitzen, ist redlich.
\item unredlicher Besitz: wer glaubt, einen anderen Rechtstitel zu besitzen, ist 
unredlich.
\end{itemize}

Körperliche Übergabe von Hand zu Hand
Übergabe durch Zeichen, wenn körperliche Übergabe nicht möglich ist z.B. Urkunden $\rightarrow$ Besitzübergabe durch Urkunde, durch Schlüsselübergabe bei Wohnung.
Übergabe durch Erklärung

Besitzverlust wenn:

\begin{itemize}
\item die Sache freiwillig aufgegeben wird
\item die Sache durch einen Erworben wurde
\item die Sache unter geht
\end{itemize}

Besitzstörungslage ist in der Zivilprozessordnung geregelt, die Klage ist binnen 30 Tagen in Kenntnis beim Bezirksgericht einzureichen. Das Eigentum ist das absolute Recht an einer Sache, der Eigentümer kann die Sache beliebig  benutzen, Verfügen, ...
\newline
\newline
Verfügungsrechte:

\begin{itemize}
\item Das Recht eine Sache zu benutzen. (Auto fahren)
\item Das Recht, Erträge, die mit der Benutzung der Sache einhergehen, zu behalten. (Auto vermieten)
\item Das Recht, die Sache in Form und Aussehen zu verändern. (Auto lackieren)
\item Das Recht, die Sache gesamt oder teilweise zu veräußern und den Veräußerungsgewinn einzubehalten. (Auto verkaufen)
\end{itemize}

\subsection{Arten von Eigentum}

\begin{itemize}
\item Alleineigentum: d.h. nur eine Person ist Verfügungsberechtigt.
\item Miteigentum: mehrere Personen sind Eigentümer einer bestimmten Sache, das Eigentumsrecht richtet sich nach einigen Quoten 
\item Gesamthandeigentum: alle Miteigentümer müssen gemeinschaftlich handeln, niemand kann alleine über seinen Anteil verfügen (z.B. Personengesellschaften)
\end{itemize}

\subsection{Eigentumserwerb}

Fund: Der Finder ist, wer eine verlorene oder vergessene Sache an sich nimmt. Der Finder hat die Sache bei der Fundbehörde abzugeben und erlangt nach einem Jahr Eigentum.
\newline
\newline
Beim Melden des Ursprünglichen Eigentümers $\rightarrow$ Finderlohn; bis 2000 Euro gibt es 10\%, darüber hinaus 5\%.
Erwerb durch Zuwachs auch die abgesonderten Früchte gehören den Eigentümer.

\subsection{Erbsitzung}

Rechtserwerb durch Zeitablauf. Voraussetzung (redlich/echt/rechtmäßig) 3 Jahre bei beweglichen Gütern, 30 Jahren bei unbeweglichen

\subsection{Eigentumsbeschränkung}

\begin{itemize}
\item die Enteignung liegt vor wenn,
\begin{itemize}
\item öffentliche Bedarf besteht (Gehweg, ...)
\item alternativen Unmöglich ist
\item Anspruch auf angemessene Entschädigung besteht
\end{itemize}
\item Forstgesetz: Wald hat der Öffentlichkeit zur Erholungszwecken zur Verfügung zu stehen
\item Denkmalschutz
\item Naturschutz
\item Immissionen: Eigentumsrechte enden dort, wo sie in andere Eigentumsrechte $\rightarrow$ friedliches Zusammenleben soll ermöglicht werden $\rightarrow$ Beschränkungen zulässig: Ortsübliches Ausmaß z.B. Lärm, Abwässer, Beleuchtung, Geruch, ...
\end{itemize}

\section{Sachrecht}
Siehe Seite 170ff\\
Ergänzung: 
\begin{itemize}
\item Innehabung: Man hat keinen Besitzwillen es geht ausschließlich darum die Sache zu verwahren bzw zu transportieren (Garderobenfrauen, Paketdienst, Postbote,...) 
\item Besitz: Ist derjenige, der die Sache willentlich besitzt und nicht bereit ist sie wieder herzugeben 
\begin{itemize}
\item rechtmäßiger Besitz: gültiger Rechtstitel (zB Vertrag (mündlich, schriftlich)) ist damit verbunden
\item unrechtmäßiger Besitz: Diebstahl
\item redlicher Besitz: Der Besitzer ist in dem glauben, dass es wirklich "in Ordnung" gegangen ist. "Wer glaubt einen gültigen Rechtstitel zu besitzen, besitzt redlich. Wer weiß, dass eine Sache jemand anderem gehört, erwirbt unredlich. zB Hehler"
\item echter Besitz: wenn Rechtmäßigkeit und Redlichkeit gegeben sind
\item unechter Besitz: Räuber und Betrüger
\item Erlangung von Besitz: körperliche Übergabe; Übergabe durch Zeichen, wenn körperliche Übergabe nicht möglich ist(Besitzerwerb durch Urkunde, Schlüsselübergabe bei Wohnung/Auto); Übergabe durch Erklärung.
\item Verlust von Besitzt: wenn die Sache freiwillig aufgegeben wird; die Sache durch einen anderen erworben wurde; die Sache untergeht..\\
\item Besitzstörungsklage: ist in der Zivilprozessordnung geregelt, die Klage ist binnen 30 Tagen ab Kenntnis beim zuständigen Bezirksgericht einzuwenden
\end{itemize}
\item Eigentum: Ist das Absolute Recht an einer Sache. Der Eigentümer kann die Sache beliebig benützen oder über sie verfügen oder sie zerstören.\\ 
Arten von Eigentum: 
\begin{itemize}
\item Alleineigentum: Nur eine Person ist Verfügungsberechtigt
\item Miteigentum: Mehrere Personen sind Eigentümer einer bestimmten Sache. Das Eigentumsrecht richtet sich nach Quoten, Rechte und Pflichten richten sich nach Quoten
\item Gesamthandeigentum: Alle Miteigentümer müssen gemeinschaftlich handeln, niemand kann alleine über seinen Anteil verfügen (zB Personengesellschaften)
\end{itemize}
Eigentumserwerb:
\begin{itemize}
\item Fund: Finder ist wer eine verlorene oder vergessene Sache entdeckt und Ansich nimmt. Der Finder hat die Sache bei der zuständigen Stelle abzugeben und erlangt nach einem Jahr Eigentum. \\
Bei Meldung des ursprünglichen Eigentümers  Finderlohn: bis 2000Euro 10\% drüber 5\%
\item Erwerb durch Zuwachs: Auch die abgesonderten Früchte gehören dem Eigentümer der Hauptsache zB Äpfel-Apfelbaum, Henne-Ei
\item Ersitzung: Rechtserwerb durch Zeitablauf (Voraussetzung: redlich echt und rechtmäßig) drei Jahre bei beweglichen Gütern 30 Jahre bei unbeweglichen 
\end{itemize}
Eigentumsbeschränkung:
\begin{itemize}
\item Enteignung: liegt vor wenn:
\begin{itemize}
\item Öffentlicher Bedarf besteht
\item eine Alternative unmöglich ist
\item Entstädigung besteht
\end{itemize}
\item Forstgesetz: Wald hat der Öffentlichkeit zur Erholung zur Verfügung zu stehen 
\item Denkmalschutz
\item Immissionen: Eigentumsrechte enden dort wo sie in die Eigentumsrechte eines anderen eingreifen $\rightarrow$ friedliches Zusammenleben soll ermöglicht werden $\rightarrow$ Beschränkungen zulässig; Ortsübliches Ausmaß zB Lärm (Hahn auf dem Land), Abwässer
\end{itemize}
\end{itemize}
\section{Widerholungsfragen}
\begin{enumerate}
\item Worin liegt der Unterschied zwischen Schuldrecht und Sachenrecht?\\
Schuldrecht: Rechtsbeziehung zwischen 2 Personen (juristischen, natürliche), Sachenrecht: Rechtsbeziehung zwischen einer Person und einer Sache, gegenüber jedermann
\item Sachen können nach unterschiedlichsten Kriterien eingeteilt werden. Suchen Sie solche Kriterien bei folgenden Sachen:\\
\begin{itemize}
\item Buch - körperlich, beweglich, vertretbar, schätzbar
\item Haus - körperlich, unbeweglich, vertretbar, schätzbar
\item Strom - unkörperlich, beweglich, vertretbar, schätzbar
\item Ölgemälde - körperlich, beweglich, unvertretbar, schätzbar
\item Levis-Jean - körperlich, beweglich, vertretbar, schätzbar
\item Coca-Cola - körperlich, beweglich, vertretbar, schätzbar
\item Autohupe - körperlich beweglich vertretbar schätzbar
\item Forderung - unkörperlich unbeweglich vertretbar schätzbar
\end{itemize}
\item Erläutern sie die Rechtsstellung der Tiere\\
Tiere fallen unter das Sachenrecht haben jedoch eine Sonderregelung sie sind unschätzbar und unvertretbar
\item Was ist, im Vergleich zu sonsitgen Gerichtsverfahren, das Besondere am Besitzstörungsverfahren\\
Wird wegen Besonderer Dringlichkeit besoners schnell abgewickelt.
\item Ist ein Eigentümer immer auch Besitzer?\\
Ein Eigentümer ist immer Besitzer aber ein Besitzer muss nicht der Eigentümer einer Sache sein. Wenn ich einen Apfel kaufe habe ich nicht vor den wieder zurückzugeben..
\item Aus welchen Gründen kann das Eigentumsrecht eingeschränkt werden?\\
Kann durch Immisionen Forstrecht Enteignung und Denkmalschutz eingeschrenkt werden.
\item Geben sie einen Überblick über die ursprünglichen Eigentumserwerbsarten.\\
Fund: Durch Finden einer Sache und nach Ablauf einer gewissen Zeit geht die Sache in das Eigentum des Finders über\\
Zuwachs: Abgesonderte Früchte meines Eigentums gehen in meinen Besitz über.\\
Ersitzung: Wenn man der redliche echte und rechtmäßig Besitzer einer Sache ist wird man nach Ablauf einer Gewissen Zeit dessen Eigentümer\\
\item Was ist bei einem abgeleiteten Eigentumserwerb immer notwendig?\\
Ein gültiger Rechtstitel und Übergabe (Eintragung ins Grundbuch bei Liegenschaften)
\item Unter welchen Vorraussetzungen kann man vom Nichteigentümer Eigentum erwerben?\\
Pfand, 
\item Wodurch können Pfandrechte entstehen?\\
Pfandleihanstalt, durch Exekution, durch Gesetzvollzug
\item Erkären Sie den Begriff Hypothek.\\
Finanzielle Belastung, Abtretung von Rechten an einem Grundstück
\item Sie tragen die Inen als Pfand gegebene Perlenkette auf dem Maturaball. Ist das erlaubt?\\
Nein. Man ist nur Innehaber der Perlenkette
\item Was haben Grunddienstbarkeiten und persöhnliche Dienstbarkeiten gemeinsam und wodurch unterscheiden sie sich?\\
-
\item Gemeinsam mit ihren 5 Geschwistern erben Sie ein Haus mit 6 gleichgroßen Wohnungen. \\
\begin{itemize}
\item Sie werden zu einem 6tel Miteigentümer des Hauses. Nehmen Sie an, an den 6 Wohnungen wurde Wohnungseigentum begründet\\
\item Jeder von ihnen erbt eine Wohnung. Worin liegt der Unterschied?\\
\end{itemize}
Im ersten Fall haben alle mitzureden beim zweiten darf ich allein entscheiden
\item Wurde in Folgenden Fällen EIgentum erworben
\begin{itemize}
\item Das Fahrad iHrer Mutter schenekn sie ihrer gutgläubigen Freundin.\\
Nein
\item Sie versetzten vertragswidrig die geliehen Perlenkette Ihrer Tante im Dorotheum. Dort kauf sie ihre gutgläubige Freundin.\\
Ja (man kann im Dorotheum davon ausgehen dass sie rechtmäßig erworben werden kann)
\item Die Ihnen als Pfand gegebene Uhr verkaufen Sie vor Fälligkeit der Forderung einem gutgläubigen Freund 
\item Ein Dieb verkauft das gestohlene Bild einem gutgläubigen Galeriebesitzar. Diesem kaufen sie es gutgläubig ab.\\
Ja
\item Sie unterschreiben den Kaufvertrag für einen PKW und bezahlen den vollen Kaufpreis. Einen Tag bevor ihnen das Auto übergeben wird wurde es gestohlen\\
Nein (es fehlt die übergabe)
\item Ein Juwelier kauft ein gestohlenes Radio und verkauft es Ihnen 
Nein (Ein Juwelier verkauft eigentlich keine Radios)
\end{itemize}
\item Eine Firma für Möbeltransporte stellt anlässlich der Übersiedlung eines Mieters einen Teil der Möbel für einige Stunden im unversperrten Vorraum der Nachbarwohnung ab. um Platz für den Transport des schweren Klaviers zu schaffen. Der Nachbar wollte in seiner Wohnung eine Party veranstalten, die natürlich abgesagt werden muss. Welche rechtlichen Maßnahmen kann er ergreifen.\\
Besitzstörungsklage
\item Ein Rechtsanwalt wird durch einen geparkten PKW an der ausfahrt aus seiner Garage gehindert. Da er verschiedene Gerichtstermine hat, muss er zahlreiche Fahren mit dem Taxi erledigen. Der Lenker des geparkten PKW erhält zwar von der Polizei eine Organstrafverfügung, doch damit ist dem Rechtsanwalt nicht geholfen. Was wird er unternehmen.\\
Schadensersatz, Besitzstörungsklage
\item Das Wochenendhäuschen war bis lang eine Insel der Ruhe und Erholung - plötzlich aber installiert der Nachbar eien Kegelbahn und hat ständig lärmende Gäste. Was kann man tun?\\
Nichts
\item An einem Wintertag dringt Schmelzwasser, anstatt durch eine Darchrinne abzufließen, in den Nachbargrund ein und verursact an den Außen und Innenwändern des Nachbarhauses erhebliche Schäden. Wie kann sich der Nachbar rechtlich zur Wehr setzen.
Er kann sofort versuchen den Schaden zu minimieren und den Nachbar auf Schadensersatz zu klagen.
\item Die 17 Jährige Eigentümerin eines Grundstückes wird unter Sachwalterschaft gestellt. Wird diese Tatsache ins Grundbuch eingetragen? Wenn ja, wie und warum?\\
ja, auf der B Seite, damit sich die anderen sicher sein können dass alles in Ordnung ist.
\item Der 70 Jährige Landwirt, geht ins Ausgedinge(die Hütte neben dem Hof) übergibt den Hof an seinen Sohn bezieht eine eigene Wohnung im Verband des Bauernhofes und unterschreibt einen Ausgedingevertrag. Ins Grundbuch wird das Ausgedinge allerdings nicht eingetragen. Nach einigen Jahren verkauft der Sohn den Bauernhof um vom Erlös Spielschulden zu begleichen. Muss der Greis nun ausziehen oder hat ihn der Käufer des Bauernhofs mitzuübernehmen.
Nein muss er nicht weil er nicht im Grundbuch steht. 
\end{enumerate}

\section{Unterscheidung von Verträgen}
Einseitig Verpflichtende Verträge:\\
Schenkung, Testament, usw
Zweiseitig Verpflichtende Verträge:\\
Kaufvertrag, Ehe
Zielschuldverhältnis:\\
Kaufverträge (nur einmalig)
Dauerschuldverhältnis:\\
Hypotheken, Alimente, Kredite, Handyvertrag (geht alles länger)
Mündlicher Vertrag:
Gleich gültig wie ein Schriftlicher Vertrag beim Vorhandensein von Zeugen. 
Schriftlicher Vertrag:
Dauerschuldverhältnisse 

\section{Versicherungen}

\begin{itemize}
\item Gesetzlich verpflichtete Versicherungen
	\begin{itemize}
	\item Sozialversicherung
		\begin{itemize}
		\item Krankenversicherung
		\item Unfallversicherung
		\item Pensionsversicherung
		\item Arbeitslosengeld
		\end{itemize}
	\end{itemize}
\item Gesetzlich freiwillige Versicherungen
	\begin{itemize}
	\item KFZ-Haftpflicht / Kaskoversicherung
	\item Individualversicherung
	\item Haushaltsversicherung / Gebäudeversicherung (Versicherung für alles was am Gebäude unbeweglich ist)
	\item Private Unfallversicherung
	\item Lebensversicherung
	\item private Krankenversicherung
	\item private Pensionsversicherung
	\item Reiseversicherung
	\end{itemize}
\end{itemize}

\subsection{Wie funktioniert eine Versicherungen?}

Alle Mitglieder der Versicherung zahlen in zeitlichen Abständen Geld ein, alle können einen Teil des Gelds beanspruchen, jedoch benötigen nicht alle das Geld.

Angestellte | Dienstnehmer | Dienstgeber
Arbeitslosen | 3 \% | 3\%
Kranken | 3,82\% | 3,82 \%
Unfall | - | 1,4\%
Pension | 10,25\% | 12,55\%
Kammerumlagerung + Wohnbauförderungsbeitrag + ... | 1\% | 1\%
GESAMT | 18,07\% | 21,78\%

\subsection{Einschub: Gehalt}

\begin{itemize}
\item unselbständig: Bruttogehalt (wäre schön) Abzüge: Sozialversicherung und Lohnsteuer, Ergebnis: Nettogehalt.
\item selbständig: Bruttogehalt Abzüge: Einkommenssteuer
\end{itemize}

\begin{tabular}{c|c|c|c}
Kündigung & AG/AN & Ohne Grund & Frist + Termin\\
Entlassung & AG & Mit Grund\\
Vorzeitiger Austritt & AN & Mit Grund
\end{tabular}



\end{document}

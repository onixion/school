\documentclass[a4paper,final]{report}
\usepackage[german]{babel}
\usepackage[utf8]{inputenc}

\title{Zweiter Weltkrieg}

\begin{document}

\chapter{ZWEITER WELTRKIEG}

Großbritannien:

	\begin{itemize}
	\item 10.5.1940 - Amtsantritt Winston Churchills
	\item Aktion Seelöwe wegen der verlorenen Luftschlacht abgeborchen
	\end{itemize}

Seekrieg:

	\begin{itemize}
	\item Schlacht im Atlantik (Hauptkriegsschauplatz); es gelingt nicht die britischen Inseln zu blockieren
	\item die teilweisen Erfolge der deutschen U-Boot Flotte sind nach dem Einsatz des Raders weitgehend beendet
	\end{itemize}
	
Der Luftkrieg 1939-1945:

	\begin{itemize}
	\item OBL: Reichsmarschall Hermann Göring
	\item Erfolge in der Blitzkriegsphase, Erfolge vor allem durch den Einsatz der Stukas; verliert in der Luftschlacht von England; Wende im Luftkrieg
	: Bau von Langstreckenbombern und Langstreckenjäger, vor allem der Einsatz des Raders
	\item 1942: Bombardierung der deutschen Städte durch die Amerikaner und Engländer mit Brand- und Sprengbomben; Febuar 1945 Angriff auf das mit
	Flüchtlingen überfüllte Dresden
	\item die legendendären V-Waffen können nicht eingesetzt werden (Vergeltungswaffen) Angriff der RAF auf Penemünde - Entwicklungszetrum der V-Waffen
	\end{itemize}

Krieg in Nordafrika 1940-1942:

	\begin{itemize}
	\item 1940 Italien erklärt England und Frankreich den Krieg September 1940 Offensive von Libyen gegen Ägypten; Jänner 1941 Aufstellung des deutschen
	Afrikakorps (Rommel 1944 Selbstmord)
	\item bei El-Alamein bleibt der Vormarsch stecken (30.6.); 13.5.1943 Kapitulation der Heeresgruppen Afrika 252.000 deutsche und italienische Gefangene;
	Nordafrika und das Mittelmeer sind verloren und damit ist die Südflanke für den Angriff auf die Festung Europa geöffnet.
	\item Italien erklärt England und Frankreich den Krieg
	\item da Mittelmeer ist daher verloren und die Südflanke ist offen $\rightarrow$ die Alliertne laden auf Sizilien, damit ist Italien verloren, 				Mussolini 
	wird gestürtzt und Italien erklärt Deutschland den Krieg
	\item es komm zu erbitterten Kämpfen zwischen den Deutschen und Italiener
	\item Mussoline flieht nach Norditalien und gründet am Gardasee die Rebublik Salo
	\item 28.04.1945 Kapitulation der Deutschen in Italien
	\item Mussolini flieht in die Schweiz $\rightarrow$ wird von Partisanen getötet (Partisanen = Truppen; meist selbst organisiert; unterstehen keinen 
	regulären Kommando)
	\end{itemize}

Russlandfeldzug:

	\begin{itemize}
	
	\item Russland war der wichtigster Krieg für Hitler, das Ziel war seinen ideologischen Hauptfeind den Marxismus zu besiegen, will sich den Erdöl im 		Kaukasus besorgen (22.06.1941, wird der Nicht-Angriffs-Packt von Hitler gebrochen)
	\item Russland wird überfallen. Der Vorstoß beginnt sehr zügig, der Vorstoß komm jedoch im Oktober ins Stocken, wegen schlechte witterung (Regen 			$\rightarrow$ Matsch)
	\item als man 50km vor Moskau steht, kommen die Truppen entgültig zum Stehen, Winter tritt ein
	\item 8.12.1941 wird der Angriff auf Russland abgebrochen, Hilter weicht mit seinen Truppen nach Stalingrad aus und will damit die Süd-Ost Flanke
	sichern (Stalingrad befindet sich Süd-Östlich von Moskau). Stalingrad wird von den Deutschen belagert. Die Deutschen kommen in eine Verkesselung
	(es kommt zu einer Kesselschlacht). Fast keine Versorgung für die deutschen Soldaten.
	\item 31.1.1943 wird die erste Kapitulationserklärung von General Paulus unterschrieben. Ergebnis ist 95 Tausend Soldaten werden zu Kriegsgefangenen
	(5000 Überlebende).
	\item die rote Armee verfolgt die verbleibenden Truppenteile. Die Deutschen werden nach Westen gedrängt.
	\item Juli und August 1943 landen die Alliierten in Sizilien
	\item 6.6.1944 landen die Alliierten in der Normandie (D-Day).
	\item 1945 spitzt sich die Lage zu
	\item zu Jahresbeginn 1945 werden die ersten Konzentrationslager eingenommen und geschlossen
	\item Hitler kapituliert nicht
	\item 30.4.1945 erschießt sich Adolf Hitler im Führerbunker in Berlin.
	\item die Leiche von Adolf Hitler wird verbrannt werden und wird von den Russen mitgenommen (dort verliert sich die Spur)
	\item Deutschland Führerlos
	\item Nachfolger  von Hitler war Admiral Dönitz
	\item 7.5.1945 unterschreibt Dönitz die Kapitualtion und der Krieg ist vorbei
	\item die Amerikaner sind sauer auf den Angriff der JApaner auf Bel Hourbor
	\item 6.8.1945 wird über Hiroshima die erste Atombombe abgeworfen
	\item 9.8.1945 wird die Zweite über Nagasaki gezündet
	
	\item Japan Kapituliert am 2.9.1945
	
	\item Bilanz des Krieges: 55 Millionen Tote, 25 Millionen getötete Russen, 6 Millionen getötete Juden
	\item es gibt keinen offizellen Friedensvertrag zwischen den Alliierten und Deutschen
	\item Deutschland und Österreich wird von den Alliierten besetzt
	\item der Westteil von Deutschland und Österreich wird von Britten, Amerikaner und Franzosen besetzt, der Ostteil wird von den Russen besetzt (der 
	Westteil wird zu DDR)
	\item Berlin und Wien werden vier-geteilt
	\item jene Gebiete, die die Sowjetunion befreit hat, die sind nicht in die Selbständigkeit überlassen worden (Polen, Ungarn, Tschechosloweakei, Bulgarien, Rumänien, 1949 DDR) (Satellitenstaaten, =Ostblock), Bündiss: Warschauer-Pakt
	\item Jugoslavien und Albanien sind Kommunistisch, werden von den Russen nicht geführt
	\item Kalter Krieg
	\end{itemize}

\newpage
\chapter{Geschichte der DDR}

Enstanden ist die DDR aus der russischen Besatzungszone Deutschlands (Berlin ist vier-geteilt).

	\begin{itemize}

		\item Westen: Amerikaner, Groß Brittanien, Frankreich
		\item Osten: Russland
	\end{itemize}
	
1949 gibt es die Blockade Westberlins durch die DDR bzw. Sowjetunion. In Westberlin gibt es drei Flughäfen und die Alliereten habe nein Jahr lang Vorräte abgeweorfen (FLugzeuge: Rosinenbomber, Candybomber). Bis zum Ende der DDR besteht Westberlin. Nach einem Jahr wird die Blockade aufgegeben. 1949 wird offizel die DDR gegründet, als eigener Staat.

\chapter{Einschub zum Thema Sozialnationalismus}

Erste systematische Vorgehen gegen die Juden in Deutschland. Bis dorthin hat man sie verbal verletzt und aus den öffentlichen Bereichen distanziert.  Genozid (=Völkermord), Juden würden dazu Shoa sagen (Deutschland, Österreich und die Alliierten: Holocaust). November Pogrom = Reichskristallnacht. In der Nacht hat man Geschäfte, Synagogen und Häuser von Juden zerstört bzw. beschädigt. 
\newline
\newline
Am 7.November 1938 wird in Paris ein deutscher Gesandter (Ernst von Rath) von Herschel Grynszpan (17 Jahre) ermordet. Schussattentat. Deutschland will zu diesem Zeitpunkt 16000 polnische Juden ausweisen. Die Polen haben sich geweigert die Juden wieder zu nehmen. Das war der Grund für dieses Attentat. Die Tat eines Einzelnen, er wird der Polizei übergeben und wieder freigelassen. Dieses Attentat wird als Beweis für die Bosheit der Juden von Hitler verwendet.
\newline
\newline
9./10.November.1938 lebten in Innsbruck 500 Juden. Davon starben nach der Nacht 4 Juden und in Innsbruck war es in der Pogromnacht im gesamten deutschen Reich am blutigsten. Juden wurden zusammengeschlagen bzw. erschlagen, die Geschäfte der Juden wurden zerstört bzw. beschädigt. Während des Übergriffes starben auch viele nicht-Juden an Selbstmord. Vertreibung der Juden war die Folge.
\newline
\newline
Man hat zeitgleich Synagogen verbrannt, Menschen ermordet und Geschäfte der Juden zerstört. Es hat wenige gegeben, die den Juden geholfen haben, doch aus Angst haben nur sehr wenige den Juden geholfen. Man hat versucht diese Aktion zu tarnen.
\newline
\newline
In Wien gab es 42 Synagogen, die alle verbrannt wurden, und auch durch die Vertreibung der Juden aus Wien sind viele neue Wohnung freigeworden.
\newline
\newline
Ab diesem Zeitpunkt ist das gewaltsame Vorgehene gegen Juden legitin (Juden dürfen geschlagen, bestohlen werden). Die jüdischen Kinder werden aus den Schulen entfernt. Alle jüdischen Beamten werden entlassen. wirtschaftliche Ende der jüdischen Kaufleute, weil die jüdischen Händler schlecht geredet wurden. Im Pass hat ein 'J' die Juden markiert. Kuriose Gesetze besagen, dass Juden den Balkon begehen, keine Haustiere halten dürfen, Wasser abgedreht usw. Die Pogronnacht war der Auslöser dieser Bewegung gegen die Juden. Juden dürften nur ohne ihr Vermögen aus dem Ausland heraus. Die Juden, die geblieben sind, mussten Abgaben von 25\% des gesamten Vermögens für die Schäden während der Pogronnacht abgeben (Sühne Abgabe).
\newline
\newline
Cornelius Gurlitt (80 Jahre) ein Kunstsammler (1400 Kunstwerke). Vater von Cornelius ist Hildebrand Gurlitt. Er hat die Werke der Juden abgekauft.
\newline
\newline
Ab der Progonnacht werden Juden dazu gedrängt Deutschland zu verlassen, man versucht die Juden loszuwerden. Jedoch müssen die Juden ihr gesamtes Vermögen da lassen.
\newline
\newline
Jänner 1942 Weinseekonferenz - dort wurde die Endlösung der Judenfrage gelöst = Ermordung aller Juden in Europa (D,Ö, alle Länder, die Hitler erobert hat) (11 Millionen Juden). Geschafft hat man 6 Millionen zu ermorden.
\newline
\newline
1966 ''Auschwitzprozesse'' - Anklage gegen die Haupttäter - niemand hat geleugnet das es passiert ist. Sie sagten, sie habe auf Befehlen gehorcht und sind nicht schuld.
\newline
\newline
Zweite Gruppe, die im Holocaust verfolgt wurden, waren die Roma und Sinti. Die Ursache für die Verfolgung ist der, dass die Roma und Sinti Außenseiter waren. Grund für die Verfolgung der Juden war hauptsächlich die Religion. Häufig werden Völker, die keinen Staat haben, verfolgt. Roma und Sinti haben bis heute keinen Nationalstaat (haben nie einen Staat besessen).
\newline
\newline
Die drei Hauptgründe Völker zu verfolgen:
\begin{itemize}
\item Religion
\item Nomaden (herumziehende Leute)
\item kein eigener Staat
\item dunkelhäutiges Volk
\end{itemize}

Heute sind die Roma und Sinti eine eigene Volksgruppe.

\newpage

\section{Zweite Österreichische Republik}

Am 5.4.1945 übernimmt Dr. Karl Renner die Verhandlungen mit den Russen mit der Wiederherstellung Österreichs (Adolf hat noch gelebt). Warum mit den Russen? - Weil viel von ihnen 
\newline
Am 13.4.1945 wird Wien durch die rote Armee eingenommen. Bereits am 27.4.1945 wird die österreichische Unabhängigkeit mit Abstimmung der Russen proklamiert.
\newline
\newline
Am 3.5.1945 ist der Krieg in Innsbruck zu Ende, die Amerikaner haben den Krieg in Tirol beendet. Die unmittelbaren Tagen waren sehr chaotisch, Kampfhandlung haben aufgehört - Soldaten sind zu Fuß nach Hause gegangen.
\newline
\newline
Die Beiden Großen Parteien SPÖ und ÖVP wollen das in Österreich nie wieder bürgerkriegsähnliche Zustände herrschen.
\newline
\newline
Im November 1945 gibt es die ersten freien Wahlen. Mitglieder der NSDAB durften bei den ersten Wahlen nicht mitwählen (waren in etwa 30\% der österreichischen Bevölkerung). Nach den ersten Wahlen haben die Parteien versucht die Stimmen der NSDAB zu bekommen.
\newline
\newline
Wahlergebnis der ersten Wahl:

\begin{itemize}
\item ÖVP 50\%
\item SPÖ 45\%
\item Kommunisten 5\%
\end{itemize}

ÖVP und SPÖ arbeiten zusammen und bilden eine Koalition. Die Parteien sorgen für Ordnung im Staat, Programme in Schulen, die unterernährte Kinder Essen bekommen.
\newline
\newline
Österreich wird besetzt:

\begin{itemize}
\item Tirol, Voradelberg wird von den Franzosen besetzt
\item Steiermark und Osttirol wurde von den Britten besetzt
\item Salzburg, Oberösterreich (südliche des Donau) wurde von der USA besetzt
\item Oberösterreich (nördlich der Donau), Burgenland und Niederösterreich wurde von den Russen besetzt
\end{itemize}

Die französische Besatzung haben keine Probleme gemacht, anders im Osten Österreich. Die russische Besatzung war sehr belastend und machte Probleme.
\newline
\newline
Die Österreicher haben versucht Südtirol doch noch an Österreich anzuschließen (1945). Es wurde lange verhandelt und schlussendlich bleibt Südtirol bei Italien. Beschluss des Südtiroler-Autonomiepaket (Steuern die in Südtirol eingenommen werden müssen nicht abgegeben werden).
\newline
\newline
1945 bis 1966 gibt es eine große Koalition (ÖVP-SPÖ). Diese beiden Parteien dominieren und ihre Ziel ist Wiederaufbau (die Wirstschaft in Gang zu bringen usw.). Marschallplanhilfe (=Finanzierung der Amerikaner). Warum bombardieren die Amerikaner Deutschland und finanzieren jetzt den Wiederaufbau? - Damit Deutschland und ihre Nachbarländer sich mit den Amerikaner verbünden, damit Demokratie in diesen Ländern entstehen und damit sie sich schneller erholen. Amerikaner haben Soldaten in Deutschland stationiert, damit nicht Russland sie besetzten. Deutschland war ein sehr wichtiger Punkt zu dieser Zeit. Besatzung bis 1955.
\newline
\newline
1956 wird die FPÖ gegründet (drittes Lager; erstes und zweiter Lager sind die SPÖ und ÖVP). SPÖ und ÖVP haben 95\% der Stimmen bei Wahlen. Ab 1956 durften die Sozial wählen, doch sie waren nicht mit den derzeitigen Parteien  zufrieden. Bis 1986 hat die FPÖ nie mehr als 5\% der Stimmen. 1986 (Jörg Haider) haben die FPÖ bis zu 25\% der Stimmen bekommen.
\newline
\newline
1966 bis 1970 gibt es zum ersten Mal für die ÖVP eine Alleinregierung (Bundeskanzler Klaus). Die ÖVP wollen keine Veränderungen, das war zur der Zeit keine gute Strategie. 1968 finden viele Proteste statt (Studenten). Der Vietnamkrieg ist im Gange und die Menschen wollen das sich Amerika nicht überall einmischt. Der Grund warum sich Amerika eingemischt hat war, damit Vietnam nicht kommunistisch wird. Amerika musst aber wieder abziehen und Vietnam wurde kommunistisch.
\newline
\newline
1968 Erfolg einer Bewegung nach Links $\rightarrow$ Studentenproteste (im westlichen Bereich).
1970 wird Parteiführer der SPÖ Bruno Kreisky. Er hat jüdische Vorfahren und hat den zweiten Weltkrieg im Exil in Schweden miterlebt. Dort hat er ein neues Stadtsystem kennengelernt $\rightarrow$ übernimmt er auf Österreich. In den Wahlen 1970 dominiert die SPÖ als stimmen stärkste Partei.
\newline
\newline
1970 bis 1971 gibt es in Österreich eine Minderheitsregierung. Nach einem Jahr gibt es Streit in der Regierung und man macht dann 1971 eine neue Wahl. Hier Gewinnt die SPÖ mit absoluter Mehrheit bis 1983. Das nutzt Bruno aus, da er jetzt Gesetze leicht einführen lassen kann.
\newline
\newline
Kreisky startet eine Bildungsoffensive $\rightarrow$ Bau von höheren Schulen. Heiratsprämie, Geburtengeld, Stipendien, gratis Schulbücher, Schülerfreifahrt, Familienbeihilfe wurde kräftig erhöht $\rightarrow$ Leute haben mehr Geld gehabt $\rightarrow$ Bruno wählen. Viele Gesellschaftliche Veränderungen $\rightarrow$ Österreich wird freier und liberaler $\rightarrow$ Gleichberechtigung von Mann und Frau. Schwangerschaftsabbruch wurde bis zur 12 Schwangerschaftswochen abtreiben.
\newline
\newline
Strafrechtsreform, die Leute sollen resozialisieren, vor allem für Jugendlichen. Bruno Kreisky verhandelt mit Yassin Arafat (Chef der PLO). Wien wird dritte UNO-sitz und Bau des Konferenzsitz.
\end{document}